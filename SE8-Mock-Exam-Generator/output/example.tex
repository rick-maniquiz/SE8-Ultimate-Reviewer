\documentclass[12pt]{article}
\usepackage[a4paper, margin=1in]{geometry}
\usepackage{titlesec}
\usepackage{hyperref}
\usepackage{parskip}
\usepackage{fancyhdr}
\usepackage{booktabs}
\usepackage{tikz}
\pagestyle{fancy}
\fancyhf{}
\rhead{FECP5 45/45}
\lhead{1Z0-808 Mock Exam}
\rfoot{\thepage}

\titleformat{\section}{\normalfont\Large\bfseries}{\thesection}{1em}{}
\titleformat{\subsection}{\normalfont\large\bfseries}{\thesubsection}{1em}{}

\title{\textbf{1Z0-808 Mock Exam}}
\author{ExamId: *insert id* \\ Items: *insert item count* \\ Duration: 2 Hours}
\date{\today}

\begin{document}

\maketitle

Which component is responsible for converting Java bytecode into machine-specific code during execution?
What is the primary purpose of the \verb|javac| command?
Which of the following is the correct and most standard signature for the main method in a Java application?
What is the file extension of a compiled Java file?
If you want to run a Java application, which of the following must be installed on the system?
Which of the following are components of the Java Development Kit (JDK)? (Choose all that apply)
Given the following code:
\begin{verbatim}
public class Welcome {
    public static void main(String[] args) {
        System.out.println("Welcome to Java");
    }
}
\end{verbatim}
What is the result of compiling and running this code?
What is the output of the following command, assuming \verb|MyApp.java| exists and contains a valid main method?
\begin{verbatim}
java MyApp.java
\end{verbatim}
You have a file named \verb|Test.java|:
\begin{verbatim}
public class test {
    public static void main(String[] args) {
        System.out.println("Test");
    }
}
\end{verbatim}
What happens when you try to compile this file with \verb|javac Test.java|?
Consider the code:
\begin{verbatim}
public class Runner {
    public static void main(String... a) {
        System.out.println(a[0]);
    }
}
\end{verbatim}
What is the result of running the class with the command: \verb|java Runner Hello World|?
What is the primary role of the Java classpath?
What is the result of the following code?
\begin{verbatim}
public class Main {
    public static void main(String[] args) {
        // My first Java program
        System.out.println("Hello");
    }
}
\end{verbatim}
Which statement best describes the relationship between JDK, JRE, and JVM?
Which of these is NOT a valid Java identifier?
A Java source file contains two classes, \verb|A| and \verb|B|. Class \verb|A| is public. What must the name of the source file be?
What happens if a semicolon is omitted at the end of a Java statement?
Given the command: \verb|java com.example.Main arg1 arg2|, what is the value of \verb|args.length| inside the main method?
Which of the following are valid declarations for the main method? (Choose all that apply)
Which statements about Java's platform independence are true? (Choose all that apply)
Which of the following are reserved keywords in Java? (Choose all that apply)


\end{document}