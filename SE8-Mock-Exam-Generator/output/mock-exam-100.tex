\documentclass[12pt]{article}
\usepackage[a4paper, margin=1in]{geometry}
\usepackage{titlesec}
\usepackage{hyperref}
\usepackage{parskip}
\usepackage{fancyhdr}
\usepackage{booktabs}
\usepackage{enumitem}
\usepackage{tikz}
\pagestyle{fancy}
\fancyhf{}
\rhead{FECP5 45/45}
\lhead{1Z0-808 Mock Exam}
\rfoot{\thepage}

\titleformat{\section}{\normalfont\Large\bfseries}{\thesection}{1em}{}
\titleformat{\subsection}{\normalfont\large\bfseries}{\thesubsection}{1em}{}

\title{\textbf{1Z0-808 Mock Exam}}
\author{ExamId: 100 \\ Items: 56 \\ Dificulty: EASY}
\date{\today}

\begin{document}

\maketitle

\begin{enumerate}[label=(\arabic*)]
\item Which component is responsible for converting Java bytecode into machine-specific code during execution?
Choose the most correct answer. 
\begin{itemize}
\item 0) JDK (Java Development Kit)
\item 1) JIT (Just-In-Time) Compiler
\item 2) javac (Java Compiler)
\item 3) JRE (Java Runtime Environment)
\end{itemize}
\item Which is the required access modifier for the `main` method to be the entry point of an application?
Choose the most correct answer. 
\begin{itemize}
\item 0) private
\item 1) protected
\item 2) public
\item 3) No access modifier is needed.
\end{itemize}
\item Which keyword is used at the beginning of a source file to place its class into a package?
Choose the most correct answer. 
\begin{itemize}
\item 0) import
\item 1) package
\item 2) classpath
\item 3) export
\end{itemize}
\item According to standard Java naming conventions, which of the following is the most appropriate name for a class that represents a user's account?
Choose the most correct answer. 
\begin{itemize}
\item 0) userAccount
\item 1) UserAccount
\item 2) \begin{verbatim}USER_ACCOUNT\end{verbatim}
\item 3) \begin{verbatim}user_account\end{verbatim}
\end{itemize}
\item Which of the following is NOT a primitive data type in Java?
Choose the most correct answer. 
\begin{itemize}
\item 0) `int`
\item 1) `String`
\item 2) `boolean`
\item 3) `char`
\end{itemize}
\item The automatic conversion of a value from a smaller primitive data type to a larger one is known as what?
Choose the most correct answer. 
\begin{itemize}
\item 0) Narrowing
\item 1) Boxing
\item 2) Casting
\item 3) Widening
\end{itemize}
\item Which of the following statements is the primary reason for the existence of wrapper classes in Java?
Choose the most correct answer. 
\begin{itemize}
\item 0) To provide a more efficient way to store primitive types.
\item 1) To allow primitive types to be treated as objects, especially for use in collections.
\item 2) To add more methods to primitive types, such as converting them to strings.
\item 3) To secure primitive data types by encapsulating them.
\end{itemize}
\item A variable declared inside a method is known as a:
Choose the most correct answer. 
\begin{itemize}
\item 0) Static variable
\item 1) Instance variable
\item 2) Local variable
\item 3) Global variable
\end{itemize}
\item What is the value of the variable `result` after the following code is executed?\n\begin{verbatim}
int result = 10 + 5 * 2;
\end{verbatim}
Choose the most correct answer. 
\begin{itemize}
\item 0) 30
\item 1) 20
\item 2) 25
\item 3) 17
\end{itemize}
\item What is the output of the following code snippet?\n\begin{verbatim}
public class Test {
    public static void main(String[] args) {
        int score = 85;
        char grade;
        if (score >= 90) {
            grade = 'A';
        } else if (score >= 80) {
            grade = 'B';
        } else {
            grade = 'C';
        }
        System.out.println(grade);
    }
}
\end{verbatim}
Choose the most correct answer. 
\begin{itemize}
\item 0) 'A'
\item 1) 'B'
\item 2) 'C'
\item 3) The code will not compile.
\end{itemize}
\item What is the output of the following code snippet?\n\begin{verbatim}
public class ForLoop {
    public static void main(String[] args) {
        for (int i = 0; i < 3; i++) {
            System.out.print(i + " ");
        }
    }
}
\end{verbatim}
Choose the most correct answer. 
\begin{itemize}
\item 0) 0 1 2 
\item 1) 0 1 2 3 
\item 2) 1 2 
\item 3) 1 2 3 
\end{itemize}
\item What is the output of the following code snippet?\n\begin{verbatim}
public class SimpleBreak {
    public static void main(String[] args) {
        for (int i = 1; i <= 5; i++) {
            if (i == 4) {
                break;
            }
            System.out.print(i);
        }
    }
}
\end{verbatim}
Choose the most correct answer. 
\begin{itemize}
\item 0) 123
\item 1) 1234
\item 2) 1235
\item 3) 12
\end{itemize}
\item Which of the following statements is true about enums in Java?
Choose the most correct answer. 
\begin{itemize}
\item 0) An enum can be instantiated using the `new` keyword.
\item 1) An enum can extend any class.
\item 2) An enum can implement an interface.
\item 3) An enum can be declared within a method.
\end{itemize}
\item What does it mean for `String` objects to be immutable in Java?
Choose the most correct answer. 
\begin{itemize}
\item 0) The `String` class is declared `final` and cannot be extended.
\item 1) The value of a `String` object cannot be changed after it is created.
\item 2) The methods of the `String` class cannot be overridden.
\item 3) String variables cannot be reassigned to a new value.
\end{itemize}
\item What is the primary difference between `StringBuilder` and `StringBuffer`?
Choose the most correct answer. 
\begin{itemize}
\item 0) `StringBuilder` is mutable, while `StringBuffer` is immutable.
\item 1) `StringBuilder` methods are synchronized, while `StringBuffer` methods are not.
\item 2) `StringBuffer` methods are synchronized, while `StringBuilder` methods are not, making `StringBuilder` faster in single-threaded environments.
\item 3) `StringBuffer` can be converted to a `String`, but `StringBuilder` cannot.
\end{itemize}
\item What is the output of the following code?\n\begin{verbatim}
public class Gadget {
    String name = "Default";

    public Gadget(String name) {
        this.name = name;
    }

    public void printName() {
        System.out.println(name);
    }

    public static void main(String[] args) {
        Gadget g = new Gadget("Phone");
        g.printName();
    }
}
\end{verbatim}
Choose the most correct answer. 
\begin{itemize}
\item 0) Default
\item 1) Phone
\item 2) The code will not compile.
\item 3) An exception is thrown at runtime.
\end{itemize}
\item Which of the following is a valid constructor declaration for a class named `Laptop`?
Choose the most correct answer. 
\begin{itemize}
\item 0) ``public void Laptop() { }``
\item 1) ``Laptop() { }``
\item 2) ``public static Laptop() { }``
\item 3) ``public new Laptop() { }``
\end{itemize}
\item Which keyword in Java is used to refer to the current object instance from within an instance method or a constructor?
Choose the most correct answer. 
\begin{itemize}
\item 0) `self`
\item 1) `current`
\item 2) `this`
\item 3) `static`
\end{itemize}
\item Which of the following statements is most accurate regarding the Java Garbage Collector (GC)?
Choose the most correct answer. 
\begin{itemize}
\item 0) Calling `System.gc()` forces the Garbage Collector to run immediately and reclaim all eligible objects.
\item 1) The Garbage Collector guarantees that an object's `finalize()` method will be called before it is deallocated.
\item 2) The Garbage Collector is a process that runs on a predictable schedule to free up memory.
\item 3) The Garbage Collector automatically frees memory occupied by objects that are no longer reachable from any active threads.
\end{itemize}
\item Which of the following is the primary goal of encapsulation in Object-Oriented Programming?
Choose the most correct answer. 
\begin{itemize}
\item 0) To allow methods to be used by many different types of objects.
\item 1) To bundle an object's data (fields) and methods together, hiding the internal state from the outside.
\item 2) To create a parent-child relationship between classes.
\item 3) To ensure all fields in a class are declared as `public` for easy access.
\end{itemize}
\item Which keyword is used in Java to specify that a class is inheriting from another class?
Choose the most correct answer. 
\begin{itemize}
\item 0) `implements`
\item 1) `inherits`
\item 2) `extends`
\item 3) `super`
\end{itemize}
\item What is the output of the following code snippet?\n\begin{verbatim}
class Animal {
    public void makeSound() {
        System.out.println("Generic Animal Sound");
    }
}

class Dog extends Animal {
    @Override
    public void makeSound() {
        System.out.println("Woof");
    }
}

public class Test {
    public static void main(String[] args) {
        Animal myAnimal = new Dog();
        myAnimal.makeSound();
    }
}
\end{verbatim}
Choose the most correct answer. 
\begin{itemize}
\item 0) Generic Animal Sound
\item 1) Woof
\item 2) The code will not compile.
\item 3) A runtime exception is thrown.
\end{itemize}
\item Which statement is true about abstract classes?
Choose the most correct answer. 
\begin{itemize}
\item 0) An abstract class can be instantiated using the `new` keyword.
\item 1) An abstract class cannot have a constructor.
\item 2) An abstract class can have both abstract and non-abstract (concrete) methods.
\item 3) A class that contains any non-abstract methods must be declared abstract.
\end{itemize}
\item What is the primary consequence of declaring a Java class with the `final` keyword?
Choose the most correct answer. 
\begin{itemize}
\item 0) All methods in the class are implicitly `private`.
\item 1) The class cannot be instantiated.
\item 2) The class cannot be extended (subclassed).
\item 3) The class cannot have any `static` members.
\end{itemize}
\item Which of the following is a correct way to declare and initialize a one-dimensional array of integers?
Choose the most correct answer. 
\begin{itemize}
\item 0) \verb|int[] arr = {1, 2, 3};|
\item 1) \verb|int arr[] = new int(3);|
\item 2) \verb|int arr[3] = {1, 2, 3};|
\item 3) \verb|int[] arr = new int[3]{1, 2, 3};|
\end{itemize}
\item Which of the following correctly creates an `ArrayList` of `String` objects?
Choose the most correct answer. 
\begin{itemize}
\item 0) \verb|ArrayList<String> list = new ArrayList[];|
\item 1) \verb|ArrayList<String> list = new ArrayList();|
\item 2) \verb|ArrayList list = new ArrayList<String>();|
\item 3) \verb|List<String> list = new List<String>();|
\end{itemize}
\item What is the primary benefit of using generics in Java?
Choose the most correct answer. 
\begin{itemize}
\item 0) They improve the runtime performance of collections.
\item 1) They allow collections to store primitive types directly.
\item 2) They enforce type safety at compile time, reducing runtime errors.
\item 3) They eliminate the need for casting entirely.
\end{itemize}
\item Which interface is used to define the natural ordering of a class?
Choose the most correct answer. 
\begin{itemize}
\item 0) `java.util.Comparator`
\item 1) `java.lang.Comparable`
\item 2) `java.util.Sortable`
\item 3) `java.lang.Orderable`
\end{itemize}
\item What is the direct superclass of the `java.lang.Exception` class?
Choose the most correct answer. 
\begin{itemize}
\item 0) `java.lang.Object`
\item 1) `java.lang.Error`
\item 2) `java.lang.RuntimeException`
\item 3) `java.lang.Throwable`
\end{itemize}
\item Which statement best describes the purpose of the `finally` block?
Choose the most correct answer. 
\begin{itemize}
\item 0) It executes only when an exception is thrown in the `try` block.
\item 1) It executes only when no exception is thrown in the `try` block.
\item 2) It is intended for code that must be executed to clean up resources, regardless of whether an exception occurs.
\item 3) It is an alternative to a `catch` block for handling all types of exceptions.
\end{itemize}
\item Which keyword is used in a method signature to declare that it might throw a particular type of exception?
Choose the most correct answer. 
\begin{itemize}
\item 0) `throw`
\item 1) `try`
\item 2) `catch`
\item 3) `throws`
\end{itemize}
\item To be used in a `try-with-resources` statement, a class must implement which interface?
Choose the most correct answer. 
\begin{itemize}
\item 0) `java.io.Serializable`
\item 1) `java.lang.AutoCloseable`
\item 2) `java.lang.Runnable`
\item 3) `java.io.Closeable` is required; `AutoCloseable` is not sufficient.
\end{itemize}
\item Which of the following is a fundamental requirement for an interface to be considered a functional interface?
Choose the most correct answer. 
\begin{itemize}
\item 0) It must be annotated with `@FunctionalInterface`.
\item 1) It must declare exactly one abstract method.
\item 2) It must not contain any default or static methods.
\item 3) It must extend `java.lang.Functional`.
\end{itemize}
\item What is the output of the following code snippet? This question tests your understanding of the immutability of `java.time` objects.
\begin{verbatim}
import java.time.LocalDate;

public class DateTest {
    public static void main(String[] args) {
        LocalDate date = LocalDate.of(2025, 8, 2);
        date.plusDays(10);
        System.out.println(date);
    }
}
\end{verbatim}
Choose the most correct answer. 
\begin{itemize}
\item 0) `2025-08-02`
\item 1) `2025-08-12`
\item 2) The code fails to compile.
\item 3) A `DateTimeException` is thrown at runtime.
\end{itemize}
\item What is the output of the following Java code?
\begin{verbatim}
import static java.lang.Math.PI;

public class Circle {
    public static void main(String[] args) {
        double radius = 2.0;
        double area = PI * radius * radius;
        System.out.printf("%.2f", area);
    }
}
\end{verbatim}
Choose the most correct answer. 
\begin{itemize}
\item 0) `3.14`
\item 1) `12.57`
\item 2) The code does not compile because `PI` is not defined.
\item 3) The code does not compile because of an incorrect import statement.
\end{itemize}
\item What is the output of the following code?
\begin{verbatim}
public class Calculator {
    public static void sum(int... numbers) {
        int total = 0;
        for (int n : numbers) {
            total += n;
        }
        System.out.println(total);
    }

    public static void main(String[] args) {
        sum(1, 2, 3);
    }
}
\end{verbatim}
Choose the most correct answer. 
\begin{itemize}
\item 0) `3`
\item 1) `6`
\item 2) The code fails to compile because `main` cannot call `sum`.
\item 3) The code fails to compile because of the `int...` syntax.
\end{itemize}
\item What is the output of the following code? This question tests the fundamental rule of passing primitive types.
\begin{verbatim}
public class PrimitiveTest {
    public static void update(int value) {
        value = 20;
    }

    public static void main(String[] args) {
        int myValue = 10;
        update(myValue);
        System.out.println(myValue);
    }
}
\end{verbatim}
Choose the most correct answer. 
\begin{itemize}
\item 0) `10`
\item 1) `20`
\item 2) The code fails to compile.
\item 3) A runtime exception is thrown.
\end{itemize}
\item Which component is responsible for converting Java bytecode into machine-specific code during execution?
Choose the most correct answer. 
\begin{itemize}
\item 0) JDK (Java Development Kit)
\item 1) JIT (Just-In-Time) Compiler
\item 2) javac (Java Compiler)
\item 3) JRE (Java Runtime Environment)
\end{itemize}
\item Which is the required access modifier for the `main` method to be the entry point of an application?
Choose the most correct answer. 
\begin{itemize}
\item 0) private
\item 1) protected
\item 2) public
\item 3) No access modifier is needed.
\end{itemize}
\item Which keyword is used at the beginning of a source file to place its class into a package?
Choose the most correct answer. 
\begin{itemize}
\item 0) import
\item 1) package
\item 2) classpath
\item 3) export
\end{itemize}
\item According to standard Java naming conventions, which of the following is the most appropriate name for a class that represents a user's account?
Choose the most correct answer. 
\begin{itemize}
\item 0) userAccount
\item 1) UserAccount
\item 2) \begin{verbatim}USER_ACCOUNT\end{verbatim}
\item 3) \begin{verbatim}user_account\end{verbatim}
\end{itemize}
\item Which of the following is NOT a primitive data type in Java?
Choose the most correct answer. 
\begin{itemize}
\item 0) `int`
\item 1) `String`
\item 2) `boolean`
\item 3) `char`
\end{itemize}
\item The automatic conversion of a value from a smaller primitive data type to a larger one is known as what?
Choose the most correct answer. 
\begin{itemize}
\item 0) Narrowing
\item 1) Boxing
\item 2) Casting
\item 3) Widening
\end{itemize}
\item Which of the following statements is the primary reason for the existence of wrapper classes in Java?
Choose the most correct answer. 
\begin{itemize}
\item 0) To provide a more efficient way to store primitive types.
\item 1) To allow primitive types to be treated as objects, especially for use in collections.
\item 2) To add more methods to primitive types, such as converting them to strings.
\item 3) To secure primitive data types by encapsulating them.
\end{itemize}
\item A variable declared inside a method is known as a:
Choose the most correct answer. 
\begin{itemize}
\item 0) Static variable
\item 1) Instance variable
\item 2) Local variable
\item 3) Global variable
\end{itemize}
\item What is the value of the variable `result` after the following code is executed?\n\begin{verbatim}
int result = 10 + 5 * 2;
\end{verbatim}
Choose the most correct answer. 
\begin{itemize}
\item 0) 30
\item 1) 20
\item 2) 25
\item 3) 17
\end{itemize}
\item What is the output of the following code snippet?\n\begin{verbatim}
public class Test {
    public static void main(String[] args) {
        int score = 85;
        char grade;
        if (score >= 90) {
            grade = 'A';
        } else if (score >= 80) {
            grade = 'B';
        } else {
            grade = 'C';
        }
        System.out.println(grade);
    }
}
\end{verbatim}
Choose the most correct answer. 
\begin{itemize}
\item 0) 'A'
\item 1) 'B'
\item 2) 'C'
\item 3) The code will not compile.
\end{itemize}
\item What is the output of the following code snippet?\n\begin{verbatim}
public class ForLoop {
    public static void main(String[] args) {
        for (int i = 0; i < 3; i++) {
            System.out.print(i + " ");
        }
    }
}
\end{verbatim}
Choose the most correct answer. 
\begin{itemize}
\item 0) 0 1 2 
\item 1) 0 1 2 3 
\item 2) 1 2 
\item 3) 1 2 3 
\end{itemize}
\item What is the output of the following code snippet?\n\begin{verbatim}
public class SimpleBreak {
    public static void main(String[] args) {
        for (int i = 1; i <= 5; i++) {
            if (i == 4) {
                break;
            }
            System.out.print(i);
        }
    }
}
\end{verbatim}
Choose the most correct answer. 
\begin{itemize}
\item 0) 123
\item 1) 1234
\item 2) 1235
\item 3) 12
\end{itemize}
\item Which of the following statements is true about enums in Java?
Choose the most correct answer. 
\begin{itemize}
\item 0) An enum can be instantiated using the `new` keyword.
\item 1) An enum can extend any class.
\item 2) An enum can implement an interface.
\item 3) An enum can be declared within a method.
\end{itemize}
\item What does it mean for `String` objects to be immutable in Java?
Choose the most correct answer. 
\begin{itemize}
\item 0) The `String` class is declared `final` and cannot be extended.
\item 1) The value of a `String` object cannot be changed after it is created.
\item 2) The methods of the `String` class cannot be overridden.
\item 3) String variables cannot be reassigned to a new value.
\end{itemize}
\item What is the primary difference between `StringBuilder` and `StringBuffer`?
Choose the most correct answer. 
\begin{itemize}
\item 0) `StringBuilder` is mutable, while `StringBuffer` is immutable.
\item 1) `StringBuilder` methods are synchronized, while `StringBuffer` methods are not.
\item 2) `StringBuffer` methods are synchronized, while `StringBuilder` methods are not, making `StringBuilder` faster in single-threaded environments.
\item 3) `StringBuffer` can be converted to a `String`, but `StringBuilder` cannot.
\end{itemize}
\item What is the output of the following code?\n\begin{verbatim}
public class Gadget {
    String name = "Default";

    public Gadget(String name) {
        this.name = name;
    }

    public void printName() {
        System.out.println(name);
    }

    public static void main(String[] args) {
        Gadget g = new Gadget("Phone");
        g.printName();
    }
}
\end{verbatim}
Choose the most correct answer. 
\begin{itemize}
\item 0) Default
\item 1) Phone
\item 2) The code will not compile.
\item 3) An exception is thrown at runtime.
\end{itemize}
\item Which of the following is a valid constructor declaration for a class named `Laptop`?
Choose the most correct answer. 
\begin{itemize}
\item 0) ``public void Laptop() { }``
\item 1) ``Laptop() { }``
\item 2) ``public static Laptop() { }``
\item 3) ``public new Laptop() { }``
\end{itemize}
\item Which keyword in Java is used to refer to the current object instance from within an instance method or a constructor?
Choose the most correct answer. 
\begin{itemize}
\item 0) `self`
\item 1) `current`
\item 2) `this`
\item 3) `static`
\end{itemize}
\item Which of the following statements is most accurate regarding the Java Garbage Collector (GC)?
Choose the most correct answer. 
\begin{itemize}
\item 0) Calling `System.gc()` forces the Garbage Collector to run immediately and reclaim all eligible objects.
\item 1) The Garbage Collector guarantees that an object's `finalize()` method will be called before it is deallocated.
\item 2) The Garbage Collector is a process that runs on a predictable schedule to free up memory.
\item 3) The Garbage Collector automatically frees memory occupied by objects that are no longer reachable from any active threads.
\end{itemize}

\end{enumerate}

\end{document}