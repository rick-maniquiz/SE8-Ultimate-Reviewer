\documentclass[12pt]{article}
\usepackage[a4paper, margin=1in]{geometry}
\usepackage{titlesec}
\usepackage{hyperref}
\usepackage{parskip}
\usepackage{fancyhdr}
\usepackage{booktabs}
\usepackage{enumitem}
\usepackage{tikz}
\pagestyle{fancy}
\fancyhf{}
\rhead{FECP5 45/45}
\lhead{ArrayList and Basic Collections}
\rfoot{\thepage}

\titleformat{\section}{\normalfont\Large\bfseries}{\thesection}{1em}{}
\titleformat{\subsection}{\normalfont\large\bfseries}{\thesubsection}{1em}{}

\title{\textbf{1Z0-808 Exam Topic Reviewer}}
\author{TopicId: 1025 \\ Topic: ArrayList and Basic Collections}
\date{\today}

\begin{document}

\maketitle
\newpage\begin{enumerate}[label=(\arabic*)]
\section*{Beyond Arrays: The Power of \texttt{ArrayList}}
Arrays are powerful but rigid. Their biggest limitation is their fixed size. To solve this, Java provides the Collections Framework, and your go-to class from this framework will be \texttt{ArrayList}. It's essentially a resizable array on steroids. For the exam, you need to understand how it differs from an array and how to use its core methods.

\subsection{Array vs. \texttt{ArrayList}}
Let's get this straight, as the exam will test the differences.
\begin{tabular}{|l|l|l|}
\hline
\textbf{Feature} & \textbf{Array} & \textbf{ArrayList} \\
\hline
Size & Fixed, defined at instantiation & Dynamic, grows as you add elements \\
\hline
Data Types & Primitives and Objects & Objects only (uses wrappers like \texttt{Integer} for primitives) \\
\hline
Type Safety & Compile-time check & Uses Generics for type safety, e.g., \texttt{ArrayList<String>} \\
\hline
Get Size & \texttt{length} property & \texttt{size()} method \\
\hline
API & Basic access via \texttt{[]} & Rich set of methods (\texttt{add}, \texttt{remove}, \texttt{contains}, etc.) \\
\hline
\end{tabular}

\subsection{Working with \texttt{ArrayList}}
To use \texttt{ArrayList}, you must import it: \texttt{import java.util.ArrayList;}.
\begin{verbatim}
// The diamond operator <> infers the type.
ArrayList<String> names = new ArrayList<>();
List<Integer> numbers = new ArrayList<>(); // Also valid, programming to the interface

names.add("Alice"); // Adds to the end
names.add(0, "Bob"); // Adds "Bob" at index 0, shifts "Alice" to index 1

System.out.println(names.get(1)); // Prints "Alice"

String oldName = names.set(0, "Bill"); // Replaces "Bob" with "Bill", returns "Bob"

System.out.println(names.size()); // Prints 2
\end{verbatim}

\subsection{The \texttt{remove()} Method Trap}
This is a classic source of confusion and a perfect exam question. \texttt{ArrayList} has two \texttt{remove} methods:
\begin{itemize}
    \item \texttt{E remove(int index)}: Removes the element at the specified position.
    \item \texttt{boolean remove(Object o)}: Removes the first occurrence of the specified element.
\end{itemize}
When you have an \texttt{ArrayList<Integer>}, the compiler gets confused. Which one should it call?
\begin{verbatim}
List<Integer> list = new ArrayList<>();
list.add(10);
list.add(20);
list.add(30);

list.remove(1); // Calls remove(int index). Removes element at index 1 (the 20).
// List is now [10, 30]

// To remove by object value, you must cast or wrap it:
list.remove(Integer.valueOf(10)); // Calls remove(Object o). Removes the 10.
// List is now [30]
\end{verbatim}
Be extremely careful with this distinction.

\subsection{Important Utility Methods}
A few other methods you must know:
\begin{itemize}
    \item \texttt{clear()}: Removes all elements.
    \item \texttt{isEmpty()}: Returns \texttt{true} if the list has no elements.
    \item \texttt{contains(Object o)}: Returns \texttt{true} if the list contains the specified element.
    \item \textbf{\texttt{Arrays.asList()}:} A handy but tricky utility.
\begin{verbatim}
String[] array = {"A", "B"};
List<String> list = Arrays.asList(array); // Creates a List view of the array
// list.remove("A"); // Throws UnsupportedOperationException!
// list.add("C");    // Also throws UnsupportedOperationException!
list.set(0, "Z");   // This is OK. The original array becomes {"Z", "B"}.
\end{verbatim}
The list returned by \texttt{Arrays.asList()} is \textbf{not} a \texttt{java.util.ArrayList}. It's a fixed-size list backed by the original array. You cannot change its size.

\section*{Key Takeaways for the 1Z0-808 Exam}
\begin{itemize}
    \item Know the key differences between Array and ArrayList (size, type, API).
    \item Memorize the core \texttt{ArrayList} methods: \texttt{add}, \texttt{get}, \texttt{set}, \texttt{remove}, \texttt{size}.
    \item Understand the \texttt{remove(index)} vs. \texttt{remove(Object)} ambiguity with \texttt{ArrayList<Integer>}.
    \item Be aware of the fixed-size, backed nature of the \texttt{List} returned from \texttt{Arrays.asList()}.
\end{itemize}
\end{enumerate}

\end{document}