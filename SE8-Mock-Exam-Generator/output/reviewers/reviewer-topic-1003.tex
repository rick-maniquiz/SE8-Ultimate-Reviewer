\documentclass[12pt]{article}
\usepackage[a4paper, margin=1in]{geometry}
\usepackage{titlesec}
\usepackage{hyperref}
\usepackage{parskip}
\usepackage{fancyhdr}
\usepackage{booktabs}
\usepackage{enumitem}
\usepackage{tikz}
\pagestyle{fancy}
\fancyhf{}
\rhead{FECP5 45/45}
\lhead{Java Coding Conventions and Javadoc}
\rfoot{\thepage}

\titleformat{\section}{\normalfont\Large\bfseries}{\thesection}{1em}{}
\titleformat{\subsection}{\normalfont\large\bfseries}{\thesubsection}{1em}{}

\title{\textbf{1Z0-808 Exam Topic Reviewer}}
\author{TopicId: 1003 \\ Topic: Java Coding Conventions and Javadoc}
\date{\today}

\begin{document}

\maketitle
\newpage\begin{enumerate}[label=(\arabic*)]
\section*{Professionalism in Code: Conventions and Documentation}
Writing code that works is only half the battle. Writing code that is readable, maintainable, and understandable by others (including your future self) is what separates an amateur from a professional. The exam won't directly ask you ``Is this good style?'', but it will use these conventions, and understanding them prevents confusion. Furthermore, knowing how comments and Javadoc work can help you spot subtle syntax errors.

\section{Java Coding Conventions}
These are the widely accepted standards for formatting Java code. While the compiler doesn't enforce most of them, your team and your sanity will thank you for following them.

\subsection{Identifier Naming Conventions}
This is the most critical convention to know. The exam often uses unconventional (but legal) naming to trip you up.
\begin{itemize}
    \item \textbf{Classes & Interfaces:} UpperCamelCase (or PascalCase). Examples: \texttt{String}, \texttt{Runnable}, \texttt{ArrayList}.
    \item \textbf{Methods & Variables:} lowerCamelCase. Examples: \texttt{toString()}, \texttt{calculatePrice()}, \texttt{userName}, \texttt{itemCount}.
    \item \textbf{Constants (static final):} All uppercase, with words separated by underscores. Examples: \texttt{Integer.MAX_VALUE}, \texttt{Math.PI}.
    \item \textbf{Packages:} All lowercase, with levels separated by dots. Examples: \texttt{java.util}, \texttt{java.time.format}.
\end{itemize}

\textbf{Exam Trap:} The compiler only cares about syntax, not style. All of the following are \textbf{100\% legal and will compile}, but they violate convention. Do not let them throw you off.
\begin{verbatim}
public class my_class { // Legal, but bad style
    void MyMethod() {}   // Legal, but bad style
    int My_Int = 10;     // Legal, but bad style
}
\end{verbatim}

\section{Comments}
Comments are ignored by the compiler. Their purpose is to explain the \textit{why} behind your code, not the \textit{what}.
\begin{itemize}
    \item \textbf{Single-line:} Starts with \texttt{//} and goes to the end of the line.
    \item \textbf{Multi-line (or Block):} Starts with \texttt{/*} and ends with \texttt{*/}. Can span multiple lines.
    \item \textbf{Javadoc:} A special multi-line comment that starts with \texttt{/**} and ends with \texttt{*/}. Used by the \texttt{javadoc} tool.
\end{itemize}

\textbf{Exam Trap:} Multi-line comments \textbf{do not nest}. The first \texttt{*/} encountered will end the comment, which can lead to compilation errors.
\begin{verbatim}
/* This is an outer comment.
   /* This is a nested comment. */  <-- This ends the outer comment!
   This text here will cause a COMPILE ERROR.
*/
\end{verbatim}

\section{Javadoc: Generating API Documentation}
The Javadoc tool (included in the JDK) parses these special comments to create professional HTML documentation for your code.

\subsection{Structure and Placement}
A Javadoc comment is placed immediately before the class, interface, method, or field it is describing.

\subsection{Common Block Tags}
These tags, identified by the \texttt{@} symbol, provide specific information.
\begin{itemize}
    \item \texttt{@param parameterName description}: Describes a method's parameter.
    \item \texttt{@return description}: Describes a method's return value.
    \item \texttt{@throws ExceptionType description}: Describes an exception that a method may throw.
    \item \texttt{@since version}: Specifies the version the feature was added (e.g., \texttt{@since 1.2}).
    \item \texttt{@author name}: Specifies the author.
    \item \texttt{@version text}: Specifies the version of the class.
\end{itemize}

\subsection{Example}
Here is a well-documented method using Javadoc.
\begin{verbatim}
/**
 * Calculates the simple interest for a given principal.
 * <p>
 * This method does not handle negative values and assumes
 * a non-compounding interest calculation.
 *
 * @param principal   The initial amount of money. Must be positive.
 * @param rate        The annual interest rate as a decimal (e.g., 0.05 for 5%).
 * @param years       The time in years. Must be non-negative.
 * @return            The calculated simple interest.
 * @throws IllegalArgumentException if principal or years are negative.
 */
public double calculateSimpleInterest(double principal, double rate, int years) {
    if (principal < 0 || years < 0) {
        throw new IllegalArgumentException("Inputs must be non-negative.");
    }
    return principal * rate * years;
}
\end{verbatim}

\subsection{Generating Documentation}
You use the \texttt{javadoc} command, which works very similarly to \texttt{javac}.
\begin{verbatim}
// Generate docs for one file into the 'docs' directory
javadoc -d docs src/com/mycorp/utils/Calculator.java
\end{verbatim}

\section{Key Takeaways for the 1Z0-808 Exam}
\begin{itemize}
    \item \textbf{Naming Conventions:} Know them cold. UpperCamelCase for types, lowerCamelCase for members, and ALL_CAPS for constants. Don't be fooled by legal-but-unconventional names.
    \item \textbf{Comment Nesting:} Remember that \texttt{/* ... */} comments do not nest.
    \item \textbf{Javadoc Syntax:} Recognize Javadoc comments (\texttt{/**...*/}) and know the purpose of the main tags: \texttt{@param}, \texttt{@return}, and \texttt{@throws}.
\end{itemize}
\end{enumerate}

\end{document}