\documentclass[12pt]{article}
\usepackage[a4paper, margin=1in]{geometry}
\usepackage{titlesec}
\usepackage{hyperref}
\usepackage{parskip}
\usepackage{fancyhdr}
\usepackage{booktabs}
\usepackage{enumitem}
\usepackage{tikz}
\pagestyle{fancy}
\fancyhf{}
\rhead{FECP5 45/45}
\lhead{Main Method and Command Line Arguments}
\rfoot{\thepage}

\titleformat{\section}{\normalfont\Large\bfseries}{\thesection}{1em}{}
\titleformat{\subsection}{\normalfont\large\bfseries}{\thesubsection}{1em}{}

\title{\textbf{1Z0-808 Exam Topic Reviewer}}
\author{TopicId: 1001 \\ Topic: Main Method and Command Line Arguments}
\date{\today}

\begin{document}

\maketitle
\newpage\begin{enumerate}[label=(\arabic*)]
\section*{Anatomy of a Basic Java Application}
Let's dissect a simple "Hello, World!" program to understand its components before diving into the main method itself.

\begin{verbatim}
// 1. Optional Package Declaration
package com.mycompany.app;

// 2. Optional Import Statements
import java.util.Date;

// 3. Class Definition
// The file MUST be named HelloWorld.java
public class HelloWorld {

    // 4. The main method - The entry point of the application
    public static void main(String[] args) {
        // 5. A statement to execute
        System.out.println("Hello, World!");
    }
}
\end{verbatim}

\begin{itemize}
    \item \textbf{Package:} Organizes your classes into a namespace. Must be the first line of code.
    \item \textbf{Import:} Brings in classes from other packages so you can use them without specifying their full name (e.g., \texttt{Date} instead of \texttt{java.util.Date}).
    \item \textbf{Class:} The blueprint for objects. A file can have multiple classes, but only \textbf{one} can be \texttt{public}, and its name must match the file name.
    \item \textbf{main method:} The special method the JVM looks for to start execution. We will focus on this now.
\end{itemize}

\section{The \texttt{main} Method: The Gateway to Your Application}
The signature of the \texttt{main} method is a very common source of exam questions. You must know it perfectly.

\subsection{The Canonical Signature}
\begin{verbatim}
public static void main(String[] args)
\end{verbatim}
\begin{itemize}
    \item \texttt{public}: It must be accessible to the JVM, which exists outside your project's scope.
    \item \texttt{static}: The method belongs to the class, not an instance of the class. The JVM calls this method without creating an object of your class first.
    \item \texttt{void}: It does not return a value.
    \item \texttt{main}: This specific name is the identifier the JVM looks for. It is case-sensitive.
    \item \texttt{String[] args}: It accepts a single argument: an array of \texttt{String} objects. These are the command-line arguments passed to your program.
\end{itemize}

\subsection{Valid Variations and Exam Traps}
The exam will test you on variations. Memorize these!
\begin{itemize}
    \item The order of modifiers \texttt{public} and \texttt{static} can be swapped: \texttt{static public void...} is \textbf{valid}.
    \item The name of the parameter array can be anything: \texttt{args}, \texttt{myArgs}, \texttt{params} are all \textbf{valid}.
    \item The array syntax can be C-style: \texttt{String args[]} is \textbf{valid}.
    \item Varargs syntax can be used: \texttt{String... args} is \textbf{valid}.
\end{itemize}

\textbf{Example Valid Signatures:}
\begin{verbatim}
public static void main(String[] args)
static public void main(String[] arguments)
public static void main(String... options)
public static void main(String commandLine[])
\end{verbatim}

\textbf{Example INVALID Signatures (Common Traps):}
\begin{verbatim}
// Not static - The JVM can't call it without an object.
public void main(String[] args)

// Wrong return type - Must be void.
public static int main(String[] args)

// Wrong method name - Must be 'main'.
public static void Main(String[] args)

// Wrong parameter type - Must be a String array.
public static void main(String args)
\end{verbatim}

\section{Command-Line Arguments}
Arguments passed after the class name on the command line are put into the \texttt{String[]} array of the \texttt{main} method.

Consider this code in \texttt{ArgTester.java}:
\begin{verbatim}
public class ArgTester {
    public static void main(String[] args) {
        // Check if any arguments were passed
        if (args.length > 0) {
            System.out.println("First argument: " + args[0]);
        } else {
            System.out.println("No arguments provided.");
        }
    }
}
\end{verbatim}

\textbf{Compilation and Execution Scenarios:}
\begin{enumerate}
    \item Run with one argument:
    \begin{verbatim}
javac ArgTester.java
java ArgTester Hello
    \end{verbatim}
    \textbf{Output:} \texttt{First argument: Hello}

    \item Run with multiple arguments (with spaces):
    \begin{verbatim}
java ArgTester "First Argument" Second
    \end{verbatim}
    \textbf{Output:} \texttt{First argument: First Argument} \\ (Note: Only \texttt{args[0]} is printed)

    \item Run with no arguments:
    \begin{verbatim}
java ArgTester
    \end{verbatim}
    \textbf{Output:} \texttt{No arguments provided.}

\end{enumerate}

\textbf{Key Points & Exam Traps:}
\begin{itemize}
    \item Arguments are separated by spaces. To treat a value with spaces as a single argument, enclose it in quotes (e.g., \texttt{"Hello World"}).
    \item The array of arguments is \textbf{never null}. If no arguments are passed, \texttt{args} is an empty array with \texttt{length == 0}.
    \item Accessing an index that doesn't exist (e.g., \texttt{args[0]} when no arguments are passed) will throw an \texttt{ArrayIndexOutOfBoundsException} at runtime. This is a classic exam trap.
\end{itemize}

\section{Key Takeaways for the 1Z0-808 Exam}
\begin{itemize}
    \item \textbf{Implicit Import:} The \texttt{java.lang} package is always imported automatically. You never need to write \texttt{import java.lang.String;}.
    \item \textbf{File Naming:} A file can only have one \texttt{public} class, and the file must be named after it (e.g., \texttt{MyClass.java}).
    \item \textbf{\texttt{main} method:} Memorize the valid signatures and the reasons behind each keyword (\texttt{public}, \texttt{static}, \texttt{void}). Be ready for tricky variations.
    \item \textbf{Command-Line Args:} Remember that \texttt{args} is an empty array (not null) if no arguments are given. Be vigilant about potential \texttt{ArrayIndexOutOfBoundsException}.
\end{itemize}
\end{enumerate}

\end{document}