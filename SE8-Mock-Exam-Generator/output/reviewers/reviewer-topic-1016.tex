\documentclass[12pt]{article}
\usepackage[a4paper, margin=1in]{geometry}
\usepackage{titlesec}
\usepackage{hyperref}
\usepackage{parskip}
\usepackage{fancyhdr}
\usepackage{booktabs}
\usepackage{enumitem}
\usepackage{tikz}
\pagestyle{fancy}
\fancyhf{}
\rhead{FECP5 45/45}
\lhead{Constructors and Initialization Blocks}
\rfoot{\thepage}

\titleformat{\section}{\normalfont\Large\bfseries}{\thesection}{1em}{}
\titleformat{\subsection}{\normalfont\large\bfseries}{\thesubsection}{1em}{}

\title{\textbf{1Z0-808 Exam Topic Reviewer}}
\author{TopicId: 1016 \\ Topic: Constructors and Initialization Blocks}
\date{\today}

\begin{document}

\maketitle
\newpage\begin{enumerate}[label=(\arabic*)]
\section*{The Birth of an Object: A Precise Sequence}
Team, we know that the \texttt{new} keyword creates an object, but the 1Z0-808 exam demands that you know the \textit{exact} sequence of events that happens during an object's creation. The questions will often feature strange-looking code with print statements in various places, and you must be able to trace the output perfectly. Let's master this sequence.

\section{Revisiting Constructors}
A constructor's job is to initialize an object's state. Let's build on what we know.

\subsection*{Constructor Overloading}
A class can have multiple constructors, as long as their parameter lists are different (in number, type, or order of parameters). This provides flexibility in how objects are created.
\begin{verbatim}
public class Shirt {
    String color;
    char size;

    // Constructor 1: No arguments
    public Shirt() {
        this.color = "White";
        this.size = 'M';
    }

    // Constructor 2: Takes a color
    public Shirt(String color) {
        this.color = color;
        this.size = 'M';
    }

    // Constructor 3: Takes color and size
    public Shirt(String color, char size) {
        this.color = color;
        this.size = size;
    }
}
\end{verbatim}

\subsection*{Constructor Chaining with \texttt{this()}}
Notice the code duplication in the constructors above. We can eliminate this by having one constructor call another using \texttt{this()}. This is called constructor chaining.

\textbf{The Rule:} A call to \texttt{this(...)} must be the \textbf{very first statement} in a constructor.

\begin{verbatim}
public class Shirt {
    String color;
    char size;

    // The "main" constructor that does the work
    public Shirt(String color, char size) {
        this.color = color;
        this.size = size;
    }

    // This constructor calls the main one, providing a default size
    public Shirt(String color) {
        this(color, 'M'); // Calls the (String, char) constructor
    }

    // This constructor calls the second one, providing a default color
    public Shirt() {
        this("White"); // Calls the (String) constructor
    }
}
\end{verbatim}

\section{Initialization Blocks}
Sometimes you need logic to run during initialization that doesn't fit well in a constructor. Java provides two special code blocks for this.

\subsection*{Instance Initializer Block}
This is a block of code written directly in the class body, enclosed in curly braces \texttt{\{...\}}. It is executed \textbf{every time an instance of the class is created}.
\begin{verbatim}
class Log {
    {
        // This instance initializer runs for every new Log object
        System.out.println("Initializing new log instance...");
    }
}
\end{verbatim}

\subsection*{Static Initializer Block}
This block is marked with the \texttt{static} keyword. It is executed only \textbf{once}, when the JVM first loads the class into memory. It runs before any static members are used and long before any instances are created.
\begin{verbatim}
class DatabaseConnection {
    static {
        // This runs only once when the class is first loaded
        System.out.println("Loading database driver...");
    }
}
\end{verbatim}

\section{The Exam-Critical Order of Initialization}
This is the key takeaway. You must memorize this order. Let's trace the creation of a \texttt{Child} object where \texttt{Child} extends \texttt{Parent}.

\textbf{Order of Events:}
\begin{enumerate}
    \item \textbf{Class Loading Phase:}
    \begin{enumerate}
        \item \textbf{Parent} class is loaded. All \texttt{static} variable declarations and \texttt{static} initializers of \texttt{Parent} are run in the order they appear.
        \item \textbf{Child} class is loaded. All \texttt{static} variable declarations and \texttt{static} initializers of \texttt{Child} are run in the order they appear.
    \end{enumerate}
    \item \textbf{Instance Creation Phase (for \texttt{new Child()}):}
    \begin{enumerate}
        \item \textbf{Parent} object part is created first. All \textit{instance} variable declarations and \textit{instance} initializers of \texttt{Parent} are run in order.
        \item The \textbf{Parent} constructor is run.
        \item \textbf{Child} object part is created. All \textit{instance} variable declarations and \textit{instance} initializers of \texttt{Child} are run in order.
        \item The \textbf{Child} constructor is run.
    \end{enumerate}
\end{enumerate}

\textbf{Example Trace:}
\begin{verbatim}
class Parent {
    static { System.out.println("1. Parent static block"); }
    { System.out.println("3. Parent instance block"); }
    Parent() { System.out.println("4. Parent constructor"); }
}
class Child extends Parent {
    static { System.out.println("2. Child static block"); }
    { System.out.println("5. Child instance block"); }
    Child() { System.out.println("6. Child constructor"); }
    public static void main(String[] args) {
        new Child();
    }
}
\end{verbatim}

\textbf{Output:}
\begin{verbatim}
1. Parent static block
2. Child static block
3. Parent instance block
4. Parent constructor
5. Child instance block
6. Child constructor
\end{verbatim}

\section{Key Takeaways for the 1Z0-808 Exam}
\begin{itemize}
    \item \textbf{Order is King:} Burn the initialization order into your memory. Statics of parent, then statics of child. Then for the instance: instance blocks of parent, constructor of parent, instance blocks of child, constructor of child.
    \item \textbf{\texttt{this()} call:} Must be the first statement in a constructor.
    \item \textbf{Static blocks run once.} Instance blocks run for every new object.
    \item When you see an initialization question, don't rush. Grab a piece of paper and trace the execution step-by-step according to the rules.
\end{itemize}
\end{enumerate}

\end{document}