\documentclass[12pt]{article}
\usepackage[a4paper, margin=1in]{geometry}
\usepackage{titlesec}
\usepackage{hyperref}
\usepackage{parskip}
\usepackage{fancyhdr}
\usepackage{booktabs}
\usepackage{enumitem}
\usepackage{tikz}
\pagestyle{fancy}
\fancyhf{}
\rhead{FECP5 45/45}
\lhead{Date and Time API (java.time)}
\rfoot{\thepage}

\titleformat{\section}{\normalfont\Large\bfseries}{\thesection}{1em}{}
\titleformat{\subsection}{\normalfont\large\bfseries}{\thesubsection}{1em}{}

\title{\textbf{1Z0-808 Exam Topic Reviewer}}
\author{TopicId: 1033 \\ Topic: Date and Time API (java.time)}
\date{\today}

\begin{document}

\maketitle
\newpage\begin{enumerate}[label=(\arabic*)]
\section*{Introduction: A New Era for Dates and Times}
For years, Java developers struggled with the old \texttt{java.util.Date} and \texttt{java.util.Calendar} APIs. They were mutable, had confusing method names, were not thread-safe, and used a 0-based month system. The 1Z0-808 exam will test your knowledge of the modern \texttt{java.time} API introduced in Java 8, which solves all these problems. Your mantra for this topic should be: \textbf{Immutability and Clarity}.

\section{The Core Classes: Human-Readable Time}
These are the main classes you'll use to represent dates and times. They are all in the \texttt{java.time} package and are timezone-agnostic.

\subsection{\texttt{LocalDate}}
Represents a date without time or a time zone (e.g., year-month-day).
\begin{itemize}
    \item \textbf{Creation:}
    \begin{verbatim}
LocalDate today = LocalDate.now(); // Current date from system clock
LocalDate myBirthday = LocalDate.of(1995, 8, 15); // Year, Month, Day
LocalDate parsedDate = LocalDate.parse("2025-12-25"); // From a String
    \end{verbatim}
    \textbf{Exam Trap:} The month is 1-based (1 for January, 12 for December). Providing an invalid value like \texttt{LocalDate.of(2025, 13, 1)} will throw a \texttt{DateTimeException} at runtime.
\end{itemize}

\subsection{\texttt{LocalTime}}
Represents a time without a date or a time zone (e.g., hour-minute-second).
\begin{itemize}
    \item \textbf{Creation:}
    \begin{verbatim}
LocalTime now = LocalTime.now(); // Current time
LocalTime meetingTime = LocalTime.of(13, 30); // 1:30 PM
LocalTime lunchTime = LocalTime.parse("12:00:00"); // From a String
    \end{verbatim}
\end{itemize}

\subsection{\texttt{LocalDateTime}}
Represents a date and time combined, but still without a time zone.
\begin{itemize}
    \item \textbf{Creation:}
    \begin{verbatim}
LocalDateTime rightNow = LocalDateTime.now(); // Current date and time
LocalDateTime event = LocalDateTime.of(2025, 8, 5, 15, 30);
// You can also combine a LocalDate and LocalTime
LocalDateTime combined = LocalDateTime.of(myBirthday, meetingTime);
    \end{verbatim}
\end{itemize}

\section{The Golden Rule: Immutability}
Every class in the \texttt{java.time} package is \textbf{immutable}. This means once an object is created, it cannot be changed. All manipulation methods (like \texttt{plusDays}, \texttt{minusHours}) return a \textbf{new} object. Failing to assign the result of these methods is a classic mistake and a frequent exam trick.

\textbf{Incorrect Usage (Common Trap):}
\begin{verbatim}
LocalDate date = LocalDate.of(2025, 1, 20);
date.plusDays(10); // This line does nothing to the 'date' variable!
System.out.println(date); // Prints 2025-01-20
\end{verbatim}

\textbf{Correct Usage:}
\begin{verbatim}
LocalDate date = LocalDate.of(2025, 1, 20);
date = date.plusDays(10); // Re-assign the new object to the variable
System.out.println(date); // Prints 2025-01-30
\end{verbatim}

\section{Representing Spans of Time: \texttt{Period} vs. \texttt{Duration}}
The exam will test you on the difference between these two classes. The key is what kind of temporal unit they represent.

\subsection{\texttt{Period}: Date-Based Amounts}
Represents a quantity of time in terms of \textbf{years, months, and days}. It's used with \texttt{LocalDate} and \texttt{LocalDateTime}.
\begin{itemize}
    \item \textbf{Creation:} \texttt{Period p = Period.of(2, 3, 10);} // 2 years, 3 months, 10 days
    \item \textbf{Usage:} \texttt{LocalDate futureDate = today.plus(p);}
    \item \textbf{Exam Trap:} The \texttt{ofYears}, \texttt{ofMonths}, \texttt{ofDays} factory methods are static and do NOT chain. The last one called wins.
    \begin{verbatim}
// This creates a period of 5 DAYS ONLY, not 1 year and 5 days!
Period trickyPeriod = Period.ofYears(1).ofDays(5);

// To create 1 year and 5 days, you must use:
Period correctPeriod = Period.of(1, 0, 5);
    \end{verbatim}
\end{itemize}

\subsection{\texttt{Duration}: Time-Based Amounts}
Represents a quantity of time in terms of \textbf{hours, minutes, seconds, and nanoseconds}. It's used with \texttt{LocalTime} and \texttt{LocalDateTime}.
\begin{itemize}
    \item \textbf{Creation:} \texttt{Duration d = Duration.ofHours(3).plusMinutes(30);}
    \item \textbf{Usage:} \texttt{LocalDateTime later = rightNow.plus(d);}
    \item \textbf{Exam Trap:} You cannot use a \texttt{Duration} with a \texttt{LocalDate}. A \texttt{LocalDate} does not have a time component. The following code throws an \texttt{UnsupportedTemporalTypeException}.
    \begin{verbatim}
LocalDate date = LocalDate.now();
Duration d = Duration.ofHours(24);
date.plus(d); // Throws Exception at runtime!
    \end{verbatim}
\end{itemize}

\section{Formatting and Parsing}
To convert \texttt{java.time} objects to and from Strings, you use \texttt{DateTimeFormatter}.

\begin{itemize}
    \item \textbf{Formatting (Object -> String):}
    \begin{verbatim}
LocalDateTime event = LocalDateTime.of(2025, 8, 5, 15, 0);

// Using a predefined standard formatter
String iso = event.format(DateTimeFormatter.ISO_DATE_TIME);
// Result: "2025-08-05T15:00:00"

// Using a custom pattern
DateTimeFormatter formatter = 
    DateTimeFormatter.ofPattern("MMMM dd, yyyy 'at' hh:mm a");
String custom = event.format(formatter);
// Result: "August 05, 2025 at 03:00 PM"
    \end{verbatim}

    \item \textbf{Parsing (String -> Object):}
    \begin{verbatim}
String input = "May 20, 2026";
DateTimeFormatter f = DateTimeFormatter.ofPattern("MMMM d, yyyy");
LocalDate parsed = LocalDate.parse(input, f);
System.out.println(parsed); // Prints 2026-05-20
    \end{verbatim}
\end{itemize}

\section*{Key Takeaways for the 1Z0-808 Exam}
\begin{itemize}
    \item \textbf{Immutability is everything.} Manipulation methods create and return new instances. Check if the result is assigned!
    \item Know the core classes: \texttt{LocalDate}, \texttt{LocalTime}, \texttt{LocalDateTime}.
    \item Understand the difference: \texttt{Period} is for date units (Y, M, D) and works with \texttt{LocalDate}. \texttt{Duration} is for time units (H, M, S) and works with \texttt{LocalTime}.
    \item Be wary of creating objects with invalid values (e.g., month 13)---this results in a runtime \texttt{DateTimeException}.
    \item Remember that \texttt{Period.ofYears(1).ofMonths(5)} creates a period of 5 months, not 1 year and 5 months.
    \item Know how to use \texttt{DateTimeFormatter} for both formatting (\texttt{format()}) and parsing (\texttt{parse()}).
\end{itemize}
\end{enumerate}

\end{document}