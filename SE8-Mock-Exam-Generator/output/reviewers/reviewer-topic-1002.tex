\documentclass[12pt]{article}
\usepackage[a4paper, margin=1in]{geometry}
\usepackage{titlesec}
\usepackage{hyperref}
\usepackage{parskip}
\usepackage{fancyhdr}
\usepackage{booktabs}
\usepackage{enumitem}
\usepackage{tikz}
\pagestyle{fancy}
\fancyhf{}
\rhead{FECP5 45/45}
\lhead{Packages, Classpath, and JARs}
\rfoot{\thepage}

\titleformat{\section}{\normalfont\Large\bfseries}{\thesection}{1em}{}
\titleformat{\subsection}{\normalfont\large\bfseries}{\thesubsection}{1em}{}

\title{\textbf{1Z0-808 Exam Topic Reviewer}}
\author{TopicId: 1002 \\ Topic: Packages, Classpath, and JARs}
\date{\today}

\begin{document}

\maketitle
\newpage\begin{enumerate}[label=(\arabic*)]
\section*{Organizing Your Code: Beyond a Single File}
Alright team, let's level up. So far, we've dealt with single Java files. In any real-world project, and certainly on the exam, you'll work with code organized into logical units. This lesson is about the nuts and bolts of that organization: packages for structure, the classpath for finding your code, and JAR files for deploying it. Mastering this is non-negotiable.

\section{Packages: The Java Filing System}
A package serves two primary purposes: organizing your classes into a manageable namespace and controlling access to them. Think of it like folders on your computer.

\subsection{Declaration and Directory Structure}
\begin{itemize}
    \item \textbf{Declaration:} You declare a class's package with the \texttt{package} keyword. This \textbf{must} be the first non-comment statement in the file.
    \begin{verbatim}
    // File: src/com/mycorp/utils/Calculator.java
    package com.mycorp.utils;

    public class Calculator { ... }
    \end{verbatim}

    \item \textbf{Directory Mapping (Crucial Exam Point):} The package name maps \textit{directly} to a directory structure. The compiler (\texttt{javac}) and runtime (\texttt{java}) enforce this rule strictly. For the package \texttt{com.mycorp.utils}, your file system must look like this:
    \begin{verbatim}
    src/
    └── com/
        └── mycorp/
            └── utils/
                └── Calculator.java
    \end{verbatim}
    A mismatch between the package name and the folder path will result in a compile-time or runtime error.
\end{itemize}

\subsection{Compiling and Running Packaged Code}
Your commands must now be aware of this structure.
\begin{itemize}
    \item \textbf{Compiling:} You should run \texttt{javac} from the root directory of your source code (e.g., the \texttt{src} folder in the example above).
    \begin{verbatim}
    // Assume we are inside the 'src' directory
    javac com/mycorp/utils/Calculator.java
    \end{verbatim}
    This creates \texttt{Calculator.class} inside \texttt{src/com/mycorp/utils/}.

    \item \textbf{A Better Way (Using -d):} To keep source and compiled files separate, use the \texttt{-d} flag to specify a destination directory.
    \begin{verbatim}
    // From inside 'src', compile into a 'bin' directory
    // The 'bin' directory is at the same level as 'src'
    javac -d ../bin com/mycorp/utils/Calculator.java
    \end{verbatim}
    This will automatically create the \texttt{com/mycorp/utils} structure inside \texttt{bin} and place \texttt{Calculator.class} there.

    \item \textbf{Running:} To run the code, you use the \textbf{Fully Qualified Class Name (FQCN)}, which is \texttt{packageName.ClassName}. You also need to tell Java where to find the compiled files.
    \begin{verbatim}
    // Assume we are in the project root (parent of 'src' and 'bin')
    // We must tell Java to look inside the 'bin' directory
    java -cp bin com.mycorp.utils.Calculator
    \end{verbatim}
\end{itemize}

\section{The CLASSPATH: Telling Java Where to Look}
The classpath is a list of directories and JAR files that the JVM searches for your compiled \texttt{.class} files. If a class isn't found on the classpath, you'll get a \texttt{ClassNotFoundException} or \texttt{NoClassDefFoundError}.
\begin{itemize}
    \item \textbf{Default:} If you don't set it, the classpath defaults to the current directory (\texttt{.}).
    \item \textbf{Setting it:} The \texttt{-cp} (or \texttt{-classpath}) flag is the standard way to set it for both \texttt{javac} and \texttt{java}.
    \item \textbf{Syntax:} Paths are separated by \texttt{;} on Windows and \texttt{:} on Linux/macOS.
    \begin{verbatim}
    // Look in the 'bin' directory AND in an external library 'libs/utils.jar'
    java -cp "bin;libs/utils.jar" com.mycorp.Main  // Windows
    java -cp "bin:libs/utils.jar" com.mycorp.Main  // Linux/macOS
    \end{verbatim}

\end{itemize}

\section{JAR Files: Bundling Your Application}
A JAR (Java ARchive) file is essentially a ZIP file that bundles all your project's \texttt{.class} files, resources, and metadata into a single distributable unit.
\begin{itemize}
    \item \textbf{Creating a JAR:} Use the \texttt{jar} tool from the JDK. The flags \texttt{c} (create) and \texttt{f} (file) are common.
    \begin{verbatim}
    // Assume 'bin' contains our compiled classes
    // Create a file called 'app.jar' with the contents of 'bin'
    jar -cf app.jar -C bin .
    \end{verbatim}
    The \texttt{-C bin .} part is a tricky but useful pattern: it means ``change directory to \texttt{bin}, then grab everything (\texttt{.}).'' This prevents the \texttt{bin} folder itself from being inside the JAR.

    \item \textbf{Making it Executable:} To run a JAR with \texttt{java -jar}, you need a \textbf{manifest file} that specifies the entry point.
    \begin{enumerate}
        \item Create a text file, e.g., \texttt{manifest.mf}, with this content (\textbf{the file MUST end with a newline character!}):
        \begin{verbatim}
        Main-Class: com.mycorp.Main
        \end{verbatim}
        \item Create the JAR using the \texttt{m} (manifest) flag.
        \begin{verbatim}
        jar -cfm app.jar manifest.mf -C bin .
        \end{verbatim}
    \end{enumerate}
    \item \textbf{Running the JAR:} Now it's simple.
    \begin{verbatim}
    java -jar app.jar
    \end{verbatim}
    When using \texttt{-jar}, the \texttt{-cp} flag is \textit{ignored}. The classpath is defined inside the JAR's manifest if needed.
\end{itemize}

\section{Key Takeaways for the 1Z0-808 Exam}
\begin{itemize}
    \item \textbf{Package vs. Path:} The package declaration \texttt{package a.b;} must correspond to the directory structure \texttt{a/b/}.
    \item \textbf{FQCN:} Run packaged classes with their full name, e.g., \texttt{java a.b.MyClass}.
    \item \textbf{Compiler/Runtime Flags:} Know \texttt{javac -d <outdir>}, \texttt{java -cp <path>}, and \texttt{java -jar <jarfile>}.
    \item \textbf{JARs:} An executable JAR requires a manifest with a \texttt{Main-Class} attribute. \texttt{java -jar} ignores the external classpath.
\end{itemize}
\end{enumerate}

\end{document}