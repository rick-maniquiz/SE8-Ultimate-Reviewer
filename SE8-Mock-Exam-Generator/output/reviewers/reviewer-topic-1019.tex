\documentclass[12pt]{article}
\usepackage[a4paper, margin=1in]{geometry}
\usepackage{titlesec}
\usepackage{hyperref}
\usepackage{parskip}
\usepackage{fancyhdr}
\usepackage{booktabs}
\usepackage{enumitem}
\usepackage{tikz}
\pagestyle{fancy}
\fancyhf{}
\rhead{FECP5 45/45}
\lhead{Encapsulation and Access Modifiers}
\rfoot{\thepage}

\titleformat{\section}{\normalfont\Large\bfseries}{\thesection}{1em}{}
\titleformat{\subsection}{\normalfont\large\bfseries}{\thesubsection}{1em}{}

\title{\textbf{1Z0-808 Exam Topic Reviewer}}
\author{TopicId: 1019 \\ Topic: Encapsulation and Access Modifiers}
\date{\today}

\begin{document}

\maketitle
\newpage\begin{enumerate}[label=(\arabic*)]
\section*{Pillar 1: Encapsulation}
Today we're tackling the first of the four great pillars of Object-Oriented Programming: \textbf{Encapsulation}. The concept is simple but powerful: we bundle the data (fields) and the methods that operate on that data into a single unit, the class. 

But encapsulation is more than just bundling. Its main goal is \textbf{data hiding}. We want to protect an object's internal state from being changed in unexpected or invalid ways by the outside world. We achieve this by hiding the implementation details and exposing only a controlled, public interface.

Think of it like the dashboard of a car. It \textit{encapsulates} the engine's complexity. You have a simple interface (a gas pedal), and you don't need to know—or mess with—the fuel injection system directly. This prevents you from accidentally breaking the engine.

\subsection*{Implementing Encapsulation in Java}
The standard strategy is straightforward:
\begin{enumerate}
    \item Declare all instance variables as \texttt{private}. This makes them inaccessible outside the class.
    \item Provide \texttt{public} methods, called \textbf{getters} (accessors) and \textbf{setters} (mutators), to read and modify the private fields.
\end{enumerate}
\begin{verbatim}
public class Employee {
    private String name;
    private double salary;

    // Getter for name
    public String getName() {
        return name;
    }

    // Setter for name
    public void setName(String name) {
        if (name != null && !name.trim().isEmpty()) {
            this.name = name;
        }
    }

    // Getter for salary
    public double getSalary() {
        return salary;
    }
    
    // A setter can contain validation logic!
    public void setSalary(double salary) {
        if (salary >= 0) { // Protects the object's state
            this.salary = salary;
        }
    }
}
\end{verbatim}

\section{Java's Access Modifiers}
Java uses four access modifiers to enforce encapsulation and control visibility. You must know these inside and out for the exam.

\begin{itemize}
    \item \texttt{public}: The least restrictive. The member is accessible from any class in any package. This is for your public API.
    \item \texttt{protected}: The member is accessible within its own package, AND to subclasses that are in \textit{different} packages. This is a common point of confusion.
    \item \texttt{default} (Package-Private): This is what you get if you specify \textbf{no modifier at all}. The member is accessible only to classes in the exact same package. Subclasses in a different package CANNOT access it.
    \item \texttt{private}: The most restrictive. The member is accessible only from within the same class file.
\end{itemize}

\subsection*{The Definitive Access Modifier Table}
Memorize this table. It's the key to dozens of potential exam questions.

\begin{tabular}{|l|c|c|c|c|}
\hline
\textbf{Modifier} & \textbf{Same Class} & \textbf{Same Package} & \textbf{Subclass (Diff. Pkg)} & \textbf{World (Diff. Pkg)}\\
\hline
\texttt{public} & Yes & Yes & Yes & Yes \\
\hline
\texttt{protected} & Yes & Yes & Yes & No \\
\hline
\texttt{default} & Yes & Yes & No & No \\
\hline
\texttt{private} & Yes & No & No & No \\
\hline
\end{tabular}

\section{Exam Traps and Nuances}
\begin{itemize}
    \item \textbf{Top-Level Classes:} A class declaration that is not nested inside another class can only be \texttt{public} or \texttt{default}. It can never be \texttt{private} or \texttt{protected}. The exam might show this invalid code.
    \item \textbf{\texttt{protected} vs. \texttt{default}:} This is the trickiest comparison. A subclass in another package can access a \texttt{protected} member of its superclass, but not a \texttt{default} member. This is a favorite exam topic.
    \item \textbf{Method Overriding:} When a subclass overrides a method from its superclass, the access modifier in the subclass must be \textbf{the same or more accessible}. For example, you can override a \texttt{protected} method with a \texttt{public} one, but you cannot override a \texttt{public} method with a \texttt{protected} one.
\end{itemize}

\section{Key Takeaways for the 1Z0-808 Exam}
\begin{itemize}
    \item Encapsulation means \textbf{data hiding}. Protect your fields by making them \texttt{private} and provide \texttt{public} getters and setters.
    \item Master the access modifier visibility table. Be able to recall it instantly.
    \item Pay special attention to the difference between \texttt{protected} and \texttt{default} access, especially in the context of inheritance across packages.
    \item Remember the rules for access modifiers on top-level classes and for overridden methods.
\end{itemize}
\end{enumerate}

\end{document}