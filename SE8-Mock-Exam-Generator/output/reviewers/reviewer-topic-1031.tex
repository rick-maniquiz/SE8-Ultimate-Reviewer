\documentclass[12pt]{article}
\usepackage[a4paper, margin=1in]{geometry}
\usepackage{titlesec}
\usepackage{hyperref}
\usepackage{parskip}
\usepackage{fancyhdr}
\usepackage{booktabs}
\usepackage{enumitem}
\usepackage{tikz}
\pagestyle{fancy}
\fancyhf{}
\rhead{FECP5 45/45}
\lhead{Try-with-Resources}
\rfoot{\thepage}

\titleformat{\section}{\normalfont\Large\bfseries}{\thesection}{1em}{}
\titleformat{\subsection}{\normalfont\large\bfseries}{\thesubsection}{1em}{}

\title{\textbf{1Z0-808 Exam Topic Reviewer}}
\author{TopicId: 1031 \\ Topic: Try-with-Resources}
\date{\today}

\begin{document}

\maketitle
\newpage\begin{enumerate}[label=(\arabic*)]
\section*{Try-with-Resources: The Modern Way to Clean Up}
One of the most common sources of bugs in older Java code is resource leaks—forgetting to close a file, a network socket, or a database connection. The traditional way to handle this was a verbose and often clumsy \texttt{finally} block. Since Java 7, we have a much more elegant and safer solution: the try-with-resources statement.

\subsection{The Old Way: A Verbose \texttt{finally}}
Let's appreciate the problem first. Look at the code required to safely read from a file before try-with-resources:
\begin{verbatim}
FileReader fr = null;
try {
    fr = new FileReader("myFile.txt");
    //... read from the file ...
} catch (IOException e) {
    e.printStackTrace();
} finally {
    if (fr != null) {
        try {
            fr.close(); // Closing can also throw an exception!
        } catch (IOException e) {
            e.printStackTrace();
        }
    }
}
\end{verbatim}
This is ugly and easy to get wrong. The nested \texttt{try-catch} inside the \texttt{finally} is particularly nasty.

\subsection{The \texttt{AutoCloseable} Interface and the New Way}
The try-with-resources statement simplifies this immensely. It works with any object whose class implements the \texttt{java.lang.AutoCloseable} interface (or its ancestor, \texttt{java.io.Closeable}).
\begin{itemize}
    \item \textbf{The Interface:} \texttt{AutoCloseable} has just one method: \texttt{void close() throws Exception;}.
    \item \textbf{The Syntax:} You declare and initialize your resource(s) inside parentheses after the \texttt{try} keyword.
\end{itemize}
\begin{verbatim}
// The new, clean way
try (FileReader fr = new FileReader("myFile.txt")) {
    //... read from the file ...
} catch (IOException e) {
    e.printStackTrace();
}
// No 'finally' block needed! The fr.close() method is called automatically.
\end{verbatim}
The compiler generates the necessary \texttt{finally} block behind the scenes to ensure \texttt{close()} is always called.

\subsection{Multiple Resources and Suppressed Exceptions}
These are key details for the exam.
\begin{itemize}
    \item \textbf{Multiple Resources:} You can declare multiple resources, separated by a semicolon. They will be closed in the \textbf{reverse order} of declaration.
\begin{verbatim}
try (FileInputStream fis = new FileInputStream("in.txt");
     FileOutputStream fos = new FileOutputStream("out.txt")) {
    // ... copy data ...
}
// fos.close() is called first, then fis.close()
\end{verbatim}
    \item \textbf{Suppressed Exceptions:} This is a huge advantage over manual \texttt{finally} blocks. What if an exception is thrown inside the \texttt{try} block, AND the \texttt{close()} method also throws an exception?
    \begin{itemize}
        \item In a manual \texttt{finally}, the second exception would hide the first one.
        \item With try-with-resources, the first exception is the one that gets propagated. The second exception (from \texttt{close()}) is "suppressed" and attached to the first one.
    \end{itemize}
    You don't lose any information. The primary cause of the error is preserved.
\end{itemize}

\section*{Key Takeaways for the 1Z0-808 Exam}
\begin{itemize}
    \item Use try-with-resources for any object that implements \texttt{AutoCloseable}.
    \item It simplifies code and prevents resource leaks.
    \item Resources are closed automatically in the reverse order of declaration.
    \item Understand that try-with-resources correctly handles suppressed exceptions, preserving the original exception, which is a major improvement over manual \texttt{finally} blocks.
\end{itemize}
\end{enumerate}

\end{document}