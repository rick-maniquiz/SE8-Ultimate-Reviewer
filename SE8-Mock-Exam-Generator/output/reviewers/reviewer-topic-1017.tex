\documentclass[12pt]{article}
\usepackage[a4paper, margin=1in]{geometry}
\usepackage{titlesec}
\usepackage{hyperref}
\usepackage{parskip}
\usepackage{fancyhdr}
\usepackage{booktabs}
\usepackage{enumitem}
\usepackage{tikz}
\pagestyle{fancy}
\fancyhf{}
\rhead{FECP5 45/45}
\lhead{Static Members and 'this' Keyword}
\rfoot{\thepage}

\titleformat{\section}{\normalfont\Large\bfseries}{\thesection}{1em}{}
\titleformat{\subsection}{\normalfont\large\bfseries}{\thesubsection}{1em}{}

\title{\textbf{1Z0-808 Exam Topic Reviewer}}
\author{TopicId: 1017 \\ Topic: Static Members and 'this' Keyword}
\date{\today}

\begin{document}

\maketitle
\newpage\begin{enumerate}[label=(\arabic*)]
\section*{Class vs. Instance: The Meaning of \texttt{static}}
Alright team, let's clarify one of the most fundamental concepts in Java: the difference between members that belong to a class (the blueprint) and members that belong to an object (the instance). This distinction is controlled by a single keyword: \texttt{static}.

\subsection*{Instance Members (Non-Static)}
Without the \texttt{static} keyword, a field or method is an \textbf{instance member}. 
\begin{itemize}
    \item \textbf{Instance Fields:} Each object gets its own separate copy. If you have 100 \texttt{Car} objects, you have 100 different \texttt{color} fields in memory.
    \item \textbf{Instance Methods:} These methods operate on the state of a specific object. They have access to the instance's fields.
\end{itemize}

\section{The \texttt{static} Keyword}
When you add the \texttt{static} keyword, the member now belongs to the \textbf{class itself}, not to any individual object.

\subsection*{Static Variables (Class Variables)}
\begin{itemize}
    \item There is only \textbf{one copy} of a static variable, and it is shared among all instances of the class.
    \item If any object modifies a static variable, the change is visible to all other objects of that class.
    \item They are accessed using the class name, e.g., \texttt{ClassName.variableName}.
\end{itemize}
\begin{verbatim}
class Car {
    String color; // Instance variable
    static int carCount = 0; // Static variable

    public Car() {
        carCount++; // Increment the shared count for each new car
    }
}

// Usage:
System.out.println("Cars created: " + Car.carCount); // Prints 0
Car c1 = new Car();
Car c2 = new Car();
System.out.println("Cars created: " + Car.carCount); // Prints 2
\end{verbatim}

\subsection*{Static Methods}
\begin{itemize}
    \item A static method is called on the class, not on an instance, e.g., \texttt{Math.random()}.
    \item \textbf{The Most Important Rule:} A static method is not associated with any particular object instance. Therefore, it \textbf{cannot access instance members} (non-static fields or methods) directly.
\end{itemize}

\begin{verbatim}
public class Calculator {
    int lastResult; // Instance field

    // Instance method - CAN access instance field 'lastResult'
    public int add(int a, int b) {
        lastResult = a + b;
        return lastResult;
    }

    // Static method - CANNOT access 'lastResult'
    public static int multiply(int a, int b) {
        // lastResult = a * b; // COMPILE ERROR! 
        return a * b;
    }
}
\end{verbatim}
Why the error? The \texttt{multiply} method is called via \texttt{Calculator.multiply(5, 10)}. It doesn't know \textit{which} object's \texttt{lastResult} field it should access. There is no associated object.

\section{The \texttt{this} Keyword: A Reference to the Current Object}
The \texttt{this} keyword is a reference to the \textbf{current object instance}. It's an implicit variable available inside any non-static method or constructor.

\subsection*{Primary Uses of \texttt{this}}
\begin{enumerate}
    \item \textbf{To resolve ambiguity} between instance variables and parameters:
    \begin{verbatim}
public Person(String name) {
    this.name = name; // this.name is the field, name is the parameter
}
    \end{verbatim}
    \item \textbf{To call another constructor} from a constructor (constructor chaining):
    \begin{verbatim}
public Person() {
    this("Unknown"); // Calls the Person(String) constructor
}
    \end{verbatim}
\end{enumerate}

\section{The Inevitable Collision: \texttt{static} and \texttt{this}}
This is a guaranteed exam concept. Since a \texttt{static} method belongs to the class and not to any object, there is no "current object" when a static method is running.

Therefore, you \textbf{cannot use the \texttt{this} keyword} from within a static context (a static method or static initializer block). Doing so will result in a compilation error: \textit{`non-static variable this cannot be referenced from a static context`}.

\begin{verbatim}
public class MyClass {
    String instanceName = "Instance";

    public void printInstanceName() {
        System.out.println(this.instanceName); // OKAY - 'this' exists here.
    }

    public static void tryToPrint() {
        // System.out.println(this.instanceName); // COMPILE ERROR!
        // 'this' does not exist in a static context.
    }

    public static void main(String[] args) {
        // tryToPrint();
    }
}
\end{verbatim}

\section{Key Takeaways for the 1Z0-808 Exam}
\begin{itemize}
    \item \textbf{\texttt{static} means "one per class".} Non-static means "one per object".
    \item \textbf{Static Access Rule:} A static method can access other static members, but \textbf{cannot} access instance members.
    \item \textbf{Instance Access Rule:} An instance method can access both instance members (using an implicit `this`) and static members.
    \item \textbf{\texttt{this} is for instances only.} It is a reference to the current object and is not available in a static context. The exam will test this directly.
    \item You can access static members through an instance variable (e.g., \texttt{c1.carCount}), but this is bad practice. The exam may show this valid code to confuse you. The correct way is to use the class name (\texttt{Car.carCount}).
\end{itemize}
\end{enumerate}

\end{document}