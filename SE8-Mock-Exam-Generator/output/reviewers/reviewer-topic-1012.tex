\documentclass[12pt]{article}
\usepackage[a4paper, margin=1in]{geometry}
\usepackage{titlesec}
\usepackage{hyperref}
\usepackage{parskip}
\usepackage{fancyhdr}
\usepackage{booktabs}
\usepackage{enumitem}
\usepackage{tikz}
\pagestyle{fancy}
\fancyhf{}
\rhead{FECP5 45/45}
\lhead{Enums}
\rfoot{\thepage}

\titleformat{\section}{\normalfont\Large\bfseries}{\thesection}{1em}{}
\titleformat{\subsection}{\normalfont\large\bfseries}{\thesubsection}{1em}{}

\title{\textbf{1Z0-808 Exam Topic Reviewer}}
\author{TopicId: 1012 \\ Topic: Enums}
\date{\today}

\begin{document}

\maketitle
\newpage\begin{enumerate}[label=(\arabic*)]
\section*{Introduction to Enums: Type-Safe Constants}
Alright team, let's talk about \texttt{enum}. Before Java 5, developers often defined constants like this:
\begin{verbatim}
public static final int SEASON_WINTER = 0;
public static final int SEASON_SPRING = 1;
// etc.
\end{verbatim}
This works, but it's not safe. What stops you from passing a value of `5` to a method expecting a season? Nothing. The compiler can't help you. Enums solve this problem by creating a special type that can only hold a fixed set of constant values. They provide compile-time type safety, which is a huge win for robust code. For the 1Z0-808 exam, you need to know not just how to declare them, but also their more advanced features, because that's where the tricky questions lie.

\section{The Basics of Enum Declaration}
An enum is a special kind of class. The simplest form looks like this:
\begin{verbatim}
public enum Season {
    WINTER, SPRING, SUMMER, FALL
}
\end{verbatim}
Here, \texttt{WINTER}, \texttt{SPRING}, \texttt{SUMMER}, and \texttt{FALL} are not just values; they are instances of the \texttt{Season} enum. They are implicitly \texttt{public}, \texttt{static}, and \texttt{final}. You don't need to, and cannot, instantiate an enum using the \texttt{new} keyword.

\subsection*{Using Enums}
You use them like any other variable. They are especially powerful in \texttt{switch} statements.
\begin{verbatim}
Season currentSeason = Season.SUMMER;

switch (currentSeason) {
    case WINTER: // Notice: No "Season.WINTER"
        System.out.println("It's cold!");
        break;
    case SUMMER:
        System.out.println("It's hot!");
        break;
    default:
        System.out.println("It's a moderate season.");
}
\end{verbatim}

\textbf{Critical Exam Trap:} Inside a \texttt{switch} statement, you refer to the enum constants directly (e.g., \texttt{WINTER}), not with their qualified name (\texttt{Season.WINTER}). Using the qualified name will cause a compilation error. The exam loves to test this.

\section{Enums with Constructors, Fields, and Methods}
This is where enums show their power and where the exam gets interesting. An enum can have instance variables, methods, and constructors, just like a regular class.

\begin{verbatim}
public enum Season {
    WINTER("Low"),         // Calls constructor with "Low"
    SPRING("Medium"),
    SUMMER("High"),
    FALL("Medium");

    private final String expectedVisitors; // An instance field

    // Constructor - must be private or package-private
    private Season(String expectedVisitors) {
        this.expectedVisitors = expectedVisitors;
    }

    // A regular method
    public void printExpectedVisitors() {
        System.out.println(expectedVisitors);
    }
}

// Usage:
Season.SUMMER.printExpectedVisitors(); // Prints "High"
\end{verbatim}

\subsection*{Key Rules for Enum Constructors}
\begin{itemize}
    \item The constructor is called once for each constant at the time the enum class is loaded. You never call it yourself.
    \item The constructor \textbf{cannot be declared \texttt{public} or \texttt{protected}}. If you don't specify an access modifier, it is implicitly \texttt{private}. The compiler will reject a \texttt{public} constructor.
    \item The list of enum constants (e.g., \texttt{WINTER("Low")}) \textbf{must} be the first thing declared in the enum body. A semicolon is required after the last constant if there are other members (fields, methods) in the enum.
\end{itemize}

\section{Essential Enum Methods}
All enums implicitly extend the abstract class \texttt{java.lang.Enum}, so they inherit its methods. The compiler also adds a few special static methods.

\begin{itemize}
    \item \texttt{public static Season[] values()}: \\
    Returns an array containing all of the enum constants in the order they are declared. You can use this for iteration:
    \begin{verbatim}
for (Season s : Season.values()) {
    System.out.println(s);
}
    \end{verbatim}
    \item \texttt{public static Season valueOf(String name)}: \\
    Returns the enum constant with the specified name. It's case-sensitive. For example, \texttt{Season.valueOf("SUMMER")} returns \texttt{Season.SUMMER}. Passing an invalid name throws an \texttt{IllegalArgumentException}.
    \item \texttt{public final String name()}: \\
    Returns the name of the constant exactly as it's declared (e.g., \texttt{Season.WINTER.name()} returns \texttt{"WINTER"}).
    \item \texttt{public final int ordinal()}: \\
    Returns the zero-based position of the constant in its declaration. \texttt{Season.WINTER.ordinal()} is 0, \texttt{Season.SPRING.ordinal()} is 1, and so on. While useful for the exam, relying on this in real-world code is fragile, as reordering the constants will change the ordinal values.
\end{itemize}

\section{Key Takeaways for the 1Z0-808 Exam}
\begin{itemize}
    \item \textbf{Comparison:} Use \texttt{==} to compare enum constants. It's safe and fast because each constant is a singleton. \texttt{.equals()} works too, but \texttt{==} is preferred.
    \item \textbf{Inheritance:} An enum \textbf{cannot extend another class} because it implicitly extends \texttt{java.lang.Enum}. However, an enum \textbf{can implement interfaces}.
    \item \textbf{Constructors:} They are implicitly \texttt{private} and are called only when the enum is initialized. You cannot invoke them with \texttt{new}.
    \item \textbf{\texttt{switch} Syntax:} Remember to use the constant name directly (e.g., \texttt{case WINTER;}) inside a switch statement.
    \item \textbf{Semicolon:} A semicolon is mandatory after the list of enum constants if the enum body contains any other members like methods or fields.
\end{itemize}
\end{enumerate}

\end{document}