\documentclass[12pt]{article}
\usepackage[a4paper, margin=1in]{geometry}
\usepackage{titlesec}
\usepackage{hyperref}
\usepackage{parskip}
\usepackage{fancyhdr}
\usepackage{booktabs}
\usepackage{enumitem}
\usepackage{tikz}
\pagestyle{fancy}
\fancyhf{}
\rhead{FECP5 45/45}
\lhead{1Z0-808 Mock Exam}
\rfoot{\thepage}

\titleformat{\section}{\normalfont\Large\bfseries}{\thesection}{1em}{}
\titleformat{\subsection}{\normalfont\large\bfseries}{\thesubsection}{1em}{}

\title{\textbf{1Z0-808 Mock Exam}}
\author{ExamId: 100 \\ Items: 56 \\ Dificulty: HARD}
\date{\today}

\begin{document}

\maketitle
\newpage\begin{enumerate}[label=(\arabic*)]
\item (questionId: 100127) Which of the following `main` method signatures will cause a `java.lang.NoSuchMethodError: main` exception at runtime, assuming the class is compiled successfully? (Choose all that apply)
Choose all the correct answer.\begin{itemize}
\item 0) `public static void main(String... args)`

\item 1) `public void main(String[] args)`

\item 2) `public static void Main(String[] args)`

\item 3) `public static void main(String args)`

\item 4) `public static int main(String[] args)`

\end{itemize}
\item (questionId: 100926) Examine this code carefully. What is the result?\n\begin{verbatim}
public class Test {
    public static void main(String[] args) {
        Integer i = 128;
        Integer j = 128;
        int k = 128;

        if (i == j) {
            System.out.print("A");
        }
        if (i == k) {
            System.out.print("B");
        }
    }
}
\end{verbatim}
Choose the most correct answer. 
\begin{itemize}
\item 0) A

\item 1) B

\item 2) AB

\item 3) No output is produced.

\end{itemize}
\item (questionId: 102427) What is the final value of \verb|sum|?\begin{verbatim}
long[][] arr = new long[2][2];
arr[0] = new long[]{1, 2};
arr[1] = arr[0];
arr[0][0] = 5;
long sum = arr[0][0] + arr[1][0];
\end{verbatim}
Choose the most correct answer. 
\begin{itemize}
\item 0) 6

\item 1) 7

\item 2) 10

\item 3) Compilation fails.

\end{itemize}
\item (questionId: 101026) What will be printed after this code executes?\n\begin{verbatim}
String[] data = {"a", "b", "c"};
int x = 0;
for(;;){
    try {
        System.out.print(data[x++]);
    } catch (ArrayIndexOutOfBoundsException e) {
        break;
    }
}
\end{verbatim}
Choose the most correct answer. 
\begin{itemize}
\item 0) abc

\item 1) ab

\item 2) a

\item 3) An infinite loop occurs.

\end{itemize}
\item (questionId: 101129) Given the following code, which statements are true? (Choose all that apply)\n\begin{verbatim}
public class Test {
    public static void main(String... args) {
        String result = "";
        loop:
        for (int i=0; i<4; i++) {
            if (i % 2 == 0) {
                continue;
            }
            switch(i) {
                case 1: result += "A"; break;
                case 3: result += "B"; break loop;
                case 5: result += "C";
            }
            result += "D";
        }
        System.out.println(result);
    }
}
\end{verbatim}
Choose all the correct answer.\begin{itemize}
\item 0) The `continue` statement is executed when `i` is 0 and 2.

\item 1) The code enters the `switch` statement when `i` is 1 and 3.

\item 2) The string `"D"` is appended to `result` exactly once.

\item 3) The `break loop;` statement is executed.

\item 4) The final output is `ABD`.

\item 5) The final output is `AB`.

\end{itemize}
\item (questionId: 102223) What is the result of attempting to access `MyDevice.NAME` in another class?\n\begin{verbatim}
interface Device {
    String NAME = "Device";
}
interface Gadget {
    String NAME = "Gadget";
}
class MyDevice implements Device, Gadget {
    // Some code
}
// In another class:
// System.out.println(MyDevice.NAME);
\end{verbatim}
Choose the most correct answer. 
\begin{itemize}
\item 0) It prints "Device".

\item 1) It prints "Gadget".

\item 2) It results in a compile-time error due to an ambiguous field.

\item 3) It prints `null`.

\end{itemize}
\item (questionId: 100029) Which of these are valid command line argument arrays in a main method signature? (Choose all that apply)
Choose all the correct answer.\begin{itemize}
\item 0) \verb|String args[]|

\item 1) \verb|String... args|

\item 2) \verb|String[] myArgs|

\item 3) \verb|String[] _args|

\item 4) \verb|String..._args|

\end{itemize}
\item (questionId: 101826) Select all lines of code after which at least one `Gadget` object becomes eligible for garbage collection.
\begin{verbatim}
class Gadget {}
public class GadgetFactory {
    static Gadget staticGadget = new Gadget(); // Line 1
    Gadget instanceGadget = new Gadget();      // Line 2

    public static void main(String[] args) {
        GadgetFactory gf = new GadgetFactory(); // Line 3
        Gadget g1 = new Gadget();               // Line 4
        gf.build(g1);
        g1 = null;                              // Line 5
        gf = null;                              // Line 6
    }

    void build(Gadget g) {
        Gadget g2 = new Gadget();               // Line 7
    } // End of build method is effectively Line 8
}
\end{verbatim}
Choose all the correct answer.\begin{itemize}
\item 0) Line 5

\item 1) Line 6

\item 2) Line 8

\item 3) The line after the `main` method completes.

\item 4) Line 3

\end{itemize}
\item (questionId: 102722) What is the result of this code?
\begin{verbatim}
Comparator<Integer> c = (i1, i2) -> i1 - i2;
List<Integer> list = Arrays.asList(Integer.MAX_VALUE, Integer.MIN_VALUE);
Collections.sort(list, c);
System.out.println(list);
\end{verbatim}
Choose the most correct answer. 
\begin{itemize}
\item 0) `[-2147483648, 2147483647]`

\item 1) `[2147483647, -2147483648]`

\item 2) An `ArithmeticException` is thrown.

\item 3) The list remains unchanged.

\end{itemize}
\item (questionId: 100723) Consider the following class. What is the outcome?\n\begin{verbatim}
public class Test {
    static {
        i = 20; // Forward reference is ok in assignment
    }
    static int i = 10;

    public static void main(String[] args) {
        System.out.println(i);
    }
}
\end{verbatim}
Choose the most correct answer. 
\begin{itemize}
\item 0) 20

\item 1) 10

\item 2) Compilation fails due to illegal forward reference.

\item 3) 0

\end{itemize}
\item (questionId: 102126) What is the result of attempting to compile this code snippet?\n\begin{verbatim}
import java.util.*;

public class GenericsTest {
    public static void main(String[] args) {
        List<String> stringList = new ArrayList<>();
        if (stringList instanceof List<Integer>) {
            System.out.println("It's a list of Integers");
        }
    }
}
\end{verbatim}
Choose the most correct answer. 
\begin{itemize}
\item 0) The code compiles and runs, but the `if` block is never executed.

\item 1) The code compiles and throws a `ClassCastException` at runtime.

\item 2) A compile-time error occurs.

\item 3) The code compiles and runs, and the `if` block is executed due to type erasure.

\end{itemize}
\item (questionId: 100421) What is the result of attempting to compile the following code snippet?
\begin{verbatim}
int i = 10;
byte b = i;
\end{verbatim}
Choose the most correct answer. 
\begin{itemize}
\item 0) It compiles successfully because the value of `i` (10) is within the range of a `byte`.

\item 1) It fails to compile because `i` is an `int` variable, and assigning it to a `byte` requires an explicit cast.

\item 2) It compiles, but will throw a runtime exception if `i` were greater than 127.

\item 3) It compiles because the compiler can determine the constant value of `i` at compile time.

\end{itemize}
\item (questionId: 101227) Which of the following are true about enums in Java? (Choose all that apply)
Choose all the correct answer.\begin{itemize}
\item 0) An enum can be a generic type, e.g., `public enum MyEnum<T> { ... }`

\item 1) Enum constants are implicitly `public`, `static`, and `final`.

\item 2) An enum can contain a `main` method and can be executed as a standalone program.

\item 3) An enum type cannot be a subtype of another enum.

\end{itemize}
\item (questionId: 101622) What is the output of this program?\n\begin{verbatim}
public class ForwardReference {
    {
        System.out.print(value + " ");
    }
    private int value = 1;
    {
        System.out.print(value + " ");
    }

    public ForwardReference() {
        System.out.print(value);
    }

    public static void main(String... args) {
        new ForwardReference();
    }
}
\end{verbatim}
Choose the most correct answer. 
\begin{itemize}
\item 0) 1 1 1

\item 1) 0 1 1

\item 2) 0 0 1

\item 3) The code fails to compile.

\end{itemize}
\item (questionId: 102022) What is the result?
\begin{verbatim}
class SuperClass {
    static String ID = "Super";
    void printID() { System.out.println(ID); }
}

class SubClass extends SuperClass {
    static String ID = "Sub";
    void printID() { System.out.println(ID); }
}

public class TestHiding {
    public static void main(String[] args) {
        SuperClass sup = new SubClass();
        System.out.println(sup.ID);
        sup.printID();
    }
}
\end{verbatim}
Choose the most correct answer. 
\begin{itemize}
\item 0) Super\nSub

\item 1) Sub\nSub

\item 2) Super\nSuper

\item 3) Sub\nSuper

\item 4) Compilation fails.

\end{itemize}
\item (questionId: 103122) An exception is thrown from a `try-with-resources` block, another from the resource's `close()` method, and a third from the `finally` block. Which exception is ultimately propagated to the caller?
Choose the most correct answer. 
\begin{itemize}
\item 0) The exception from the `try` block.

\item 1) The exception from the `close()` method.

\item 2) The exception from the `finally` block.

\item 3) A new wrapper exception containing all three.

\end{itemize}
\item (questionId: 100424) What value is stored in the variable `result` after this code is executed?
\begin{verbatim}
long result = 2_147_483_647 + 1;
\end{verbatim}
Choose the most correct answer. 
\begin{itemize}
\item 0) `2147483648`

\item 1) `-2147483648`

\item 2) The code fails to compile.

\item 3) `21474836471`

\end{itemize}
\item (questionId: 103452) What is the output of the following code, which uses a statically imported nested class?
\begin{verbatim}
// File: Encloser.java
public class Encloser {
    public static class Nested {
        public void hi() { System.out.println("Hi"); }
    }
}

// File: Main.java
import static Encloser.Nested;

public class Main {
    public static void main(String[] args) {
        Nested n = new Nested();
        n.hi();
    }
}
\end{verbatim}
Choose the most correct answer. 
\begin{itemize}
\item 0) `Hi`

\item 1) The code fails to compile because you cannot statically import a class.

\item 2) The code fails to compile because `Nested` must be instantiated via `Encloser.Nested`.

\item 3) The code fails to compile for a different reason.

\end{itemize}
\item (questionId: 100722) What will the following code print?\n\begin{verbatim}
public class ScopePuzzle {
    int x = 5;

    public static void main(String[] args) {
        ScopePuzzle p = new ScopePuzzle();
        p.go();
    }

    void go() {
        int x;
        go2();
        // System.out.println(x); // Line X
    }

    void go2() {
        x = 10;
    }
}
\end{verbatim}
Choose the most correct answer. 
\begin{itemize}
\item 0) If Line X is uncommented, the code will print 10.

\item 1) If Line X is uncommented, the code will print 5.

\item 2) If Line X is uncommented, the code will fail to compile.

\item 3) The code as is will compile and run without error.

\end{itemize}
\item (questionId: 102320) What is the output of the following code?\n\begin{verbatim}
public class Finalizer {
    private final int value;
    public Finalizer(int v) {
        this.value = v;
    }
    public int getValue() {
        return this.value;
    }
    public static void main(String[] args) {
        final Finalizer f = new Finalizer(20);
        // Line X
        System.out.println(f.getValue());
    }
    public void modify(Finalizer fin) {
        fin = new Finalizer(30);
    }
}
\end{verbatim}
What would happen if `modify(f);` was inserted at 'Line X'?
Choose the most correct answer. 
\begin{itemize}
\item 0) The code would fail to compile because `f` is final.

\item 1) The code would print 30.

\item 2) The code would print 20.

\item 3) The code would throw a runtime exception.

\end{itemize}
\item (questionId: 100924) What are the final values of `x` and `y` after this code snippet runs?\n\begin{verbatim}
int x = 10;
int y = 20;
if (++x <= 10 && --y > 15) {
    x++;
    y++;
}
\end{verbatim}
Choose the most correct answer. 
\begin{itemize}
\item 0) `x` is 11, `y` is 20

\item 1) `x` is 11, `y` is 19

\item 2) `x` is 12, `y` is 20

\item 3) `x` is 10, `y` is 20

\end{itemize}
\item (questionId: 101627) What is the result of compiling this class? \n\begin{verbatim}
public class FinalChallenge {
    private final int value;

    public FinalChallenge() {
        this(10);
        // value = 20; // Line A
    }

    public FinalChallenge(int value) {
        this.value = value;
    }
}
\end{verbatim}
Choose the most correct answer. 
\begin{itemize}
\item 0) The code compiles successfully as is.

\item 1) The code fails to compile because a final field is assigned in one constructor but not the other.

\item 2) If Line A is uncommented, the code will fail to compile.

\item 3) The code fails to compile because a final field cannot be assigned in a constructor that uses `this()`.

\end{itemize}
\item (questionId: 101729) Given `public class Test { static int x = 1; int y = 2; }`, which of the following lines of code are valid if placed inside the `main` method of another class? (Choose all that apply)
Choose all the correct answer.\begin{itemize}
\item 0) `System.out.println(Test.x);`

\item 1) `System.out.println(Test.y);`

\item 2) `Test t = new Test(); System.out.println(t.x);`

\item 3) `Test t = new Test(); System.out.println(t.y);`

\item 4) `Test t = null; System.out.println(t.x);`

\item 5) `Test t = null; System.out.println(t.y);`

\end{itemize}
\item (questionId: 100524) What is the final value of `s`?
\begin{verbatim}
short s = 32767;
s++;
\end{verbatim}
Choose the most correct answer. 
\begin{itemize}
\item 0) `32768`

\item 1) `-32768`

\item 2) `0`

\item 3) The code does not compile.

\end{itemize}
\item (questionId: 100827) Which statement best describes the evaluation of the following expression?\n\begin{verbatim}
int a = 1, b = 2, c = 3, d = 4;
int result = a + b * c / d > a ? b + c : d - a;
\end{verbatim}
Choose the most correct answer. 
\begin{itemize}
\item 0) The expression evaluates to 5.

\item 1) The expression evaluates to 3.

\item 2) The multiplication `b*c` is performed first.

\item 3) The ternary operator `? :` has higher precedence than `>`.

\end{itemize}
\item (questionId: 101924) Given the code:
\begin{verbatim}
// In package company.parts
package company.parts;
public class Engine {
    // package-private constructor
    Engine() {}
}

// In package company.parts
package company.parts;
public class PartsFactory {
    public static Engine getEngine() {
        return new Engine();
    }
}

// In package company.vehicles
package company.vehicles;
import company.parts.*;
public class Car {
    public static void main(String[] args) {
        Engine e = PartsFactory.getEngine(); // Line X
        System.out.println("Engine acquired");
    }
}
\end{verbatim}
What is the result?
Choose the most correct answer. 
\begin{itemize}
\item 0) Compilation fails at Line X because `Engine`'s constructor is not visible.

\item 1) Compilation fails at Line X because the `Engine` class is not visible.

\item 2) Compilation succeeds, and "Engine acquired" is printed.

\item 3) Compilation fails because `PartsFactory.getEngine()` returns a type whose constructor is not public.

\end{itemize}
\item (questionId: 101528) Which statements are true regarding the initialization of a new object? (Choose all that apply)
Choose all the correct answer.\begin{itemize}
\item 0) The constructor body is executed before instance initializers.

\item 1) If present, a call to another constructor using `this()` must be the very first statement in a constructor.

\item 2) Static variables are initialized after the constructor completes.

\item 3) Instance variables are assigned their default values (e.g., 0, false, null) before any instance initializers or constructors are run.

\item 4) Instance initializers are executed in the order they appear in the source code.

\item 5) It is valid for a class to have multiple instance initializer blocks.

\end{itemize}
\item (questionId: 102520) What is the result of executing the following code?\begin{verbatim}
import java.util.List;
import java.util.ArrayList;

public class Test {
    public static void main(String[] args) {
        List<Integer> list = new ArrayList<>();
        list.add(1);
        list.add(2);
        list.add(3);
        list.remove(2);
        System.out.println(list);
    }
}
\end{verbatim}
Choose the most correct answer. 
\begin{itemize}
\item 0) [1, 2]

\item 1) [1, 3]

\item 2) [2, 3]

\item 3) An `IndexOutOfBoundsException` is thrown.

\end{itemize}
\item (questionId: 101425) Which line of code, when inserted at `// INSERT`, will result in both `boolean` variables being `true`?
\begin{verbatim}
StringBuilder sb1 = new StringBuilder("A");
StringBuilder sb2 = new StringBuilder("A");
String s1 = new String("A");

// INSERT

boolean b1 = sb1.toString().equals(s1);
boolean b2 = sb1 == sb2;
\end{verbatim}
Choose the most correct answer. 
\begin{itemize}
\item 0) `sb2 = sb1;`

\item 1) `sb1 = new StringBuilder(s1);`

\item 2) `s1 = sb1.toString(); sb2 = sb1;`

\item 3) It's impossible to make both `true`.

\end{itemize}
\item (questionId: 102928) What is the final output of this program?
\begin{verbatim}
public class Test {
    public static void main(String[] args) {
        try {
            System.out.print("A");
            danger();
        } catch (Exception e) {
            System.out.print("B");
        } finally {
            System.out.print("C");
        }
    }
    static void danger() {
        try {
            throw new Error();
        } finally {
            System.out.print("D");
        }
    }
}
\end{verbatim}
Choose the most correct answer. 
\begin{itemize}
\item 0) `ADBC`

\item 1) `ADC`

\item 2) `AD` followed by an `Error` being thrown.

\item 3) `A` followed by an `Error` being thrown.

\end{itemize}
\item (questionId: 100125) Consider the following code:
\begin{verbatim}
package com.test;
public class Runner {
    public static void main(String[] args) {
        System.out.println("OK");
    }
}
\end{verbatim}
After compiling with `javac -d . com/test/Runner.java`, you are in the `com/test` directory. You execute `java Runner`. What is the result?
Choose the most correct answer. 
\begin{itemize}
\item 0) It prints "OK".

\item 1) A `ClassNotFoundException` is thrown.

\item 2) A `NoClassDefFoundError` is thrown.

\item 3) A `SecurityException` is thrown.

\end{itemize}
\item (questionId: 100227) Which of the following statements about `import` declarations are true? (Choose all that apply)
Choose all the correct answer.\begin{itemize}
\item 0) `import` statements are required to use any class outside the current package.

\item 1) A static import can import all static members of a class using a wildcard (`*`).

\item 2) Importing a package, such as `java.util.*`, also imports its subpackages, like `java.util.concurrent`.

\item 3) Importing a class with the same simple name from two different packages requires one of them to be referred to by its fully qualified name.

\item 4) `import` statements increase the size of the final `.class` file.

\end{itemize}
\item (questionId: 100327) What is the result of compiling and running this code?
\begin{verbatim}
public class TrickyScope {
    public static void main(String[] args) {
        int i = 0;
        if (true) {
            // The following comment looks like it closes the block
            /*
                System.out.println("Inside comment");
            }
            */
            i = 1;
        } 
        System.out.println(i);
    }
}
\end{verbatim}
Choose the most correct answer. 
\begin{itemize}
\item 0) It fails to compile due to a syntax error with braces.

\item 1) It compiles and prints `0`.

\item 2) It compiles and prints `1`.

\item 3) It compiles but throws a runtime exception.

\end{itemize}
\item (questionId: 103229) Which of the following functional interface declarations will compile successfully? (Choose all that apply)
Choose all the correct answer.\begin{itemize}
\item 0) `@FunctionalInterface interface A { int m(); default int n() {return 0;} }`

\item 1) `@FunctionalInterface interface B extends A { }`

\item 2) `@FunctionalInterface interface C { <T> T m(T t); }`

\item 3) `@FunctionalInterface interface D extends java.util.Comparator { }`

\item 4) `@FunctionalInterface interface E { void m(); String toString(); }`

\end{itemize}
\item (questionId: 101529) You have an encapsulated `MutableDate` class. Which of the following getter method implementations for a `Person` class would risk breaking the encapsulation of the `Person` object's state? (Choose all that apply)\n\begin{verbatim}
// Assume MutableDate is a class like java.util.Date
// with public methods to change its state.
class MutableDate { /* ... setters ... */ }

class Person {
    private String name;
    private MutableDate birthDate;

    public Person(String name, MutableDate birthDate) {
        this.name = name;
        this.birthDate = birthDate;
    }

    // ... getters ...
}
\end{verbatim}
Choose all the correct answer.\begin{itemize}
\item 0) `public MutableDate getBirthDate() { return this.birthDate; }`

\item 1) `public String getName() { return this.name; }`

\item 2) `public MutableDate getBirthDate() { return new MutableDate(this.birthDate.getTime()); }`

\item 3) `public MutableDate getBirthDate() { return (MutableDate) this.birthDate.clone(); }` (Assume `clone()` is implemented correctly for a deep copy).

\item 4) `public void printBirthDate() { System.out.println(this.birthDate); }`

\end{itemize}
\item (questionId: 100823) What is the result of this code snippet?\n\begin{verbatim}
int mask = 0x000F;
int value = 0x2222;
System.out.println(value & mask);
\end{verbatim}
Choose the most correct answer. 
\begin{itemize}
\item 0) 15

\item 1) 2

\item 2) 0

\item 3) 2222

\end{itemize}
\item (questionId: 101327) Which statements are true about string concatenation using the `+` operator in a loop? (Choose all that apply)
\begin{verbatim}
String result = "";
for (int i=0; i<100; i++) {
    result += i; // Line 3
}
\end{verbatim}
Choose all the correct answer.\begin{itemize}
\item 0) A new `String` object is created in each iteration of the loop.

\item 1) The compiler automatically replaces this code with `StringBuilder` for efficiency.

\item 2) This is the most memory-efficient way to build a string.

\item 3) After the loop, the original `result` object (the empty string) has been modified to contain the final value.

\end{itemize}
\item (questionId: 101828) Analyze the following code. At Point Y, how many `java.lang.String` objects are eligible for GC, assuming no string pooling optimizations for literals?
\begin{verbatim}
public class StringGC {
    public static void main(String[] args) {
        String s1 = "one";
        String s2 = new String("two");
        String s3 = "three";
        s3 = s1;
        s1 = s2;
        s2 = null;

        // What about the object referred to by s1 originally ("one")?
        // What about the object referred to by s2 originally ("two")?
        // What about the object referred to by s3 originally ("three")?
        // Point Y
    }
}
\end{verbatim}
Choose the most correct answer. 
\begin{itemize}
\item 0) 0

\item 1) 1

\item 2) 2

\item 3) 3

\end{itemize}
\item (questionId: 103022) What is the result of attempting to compile and run the following code?
\begin{verbatim}
public class StaticFail {
    static {
        if (true) {
            throw new RuntimeException("Initialization failed");
        }
    }

    public static void main(String[] args) {
        System.out.println("Hello");
    }
}
\end{verbatim}
Choose the most correct answer. 
\begin{itemize}
\item 0) The code compiles and prints `Hello`.

\item 1) The code does not compile.

\item 2) The code compiles, but throws a `RuntimeException` when run.

\item 3) The code compiles, but throws an `ExceptionInInitializerError` when run.

\item 4) The code compiles, but throws a `NoClassDefFoundError` when run.

\end{itemize}
\item (questionId: 101325) What is the output of the following code?
\begin{verbatim}
String text = "a.b.c";
String[] parts = text.split(".");
System.out.println(parts.length);
\end{verbatim}
Choose the most correct answer. 
\begin{itemize}
\item 0) 0

\item 1) 1

\item 2) 3

\item 3) An exception is thrown at runtime.

\end{itemize}
\item (questionId: 103653) What is the output of this code which passes and returns references?
\begin{verbatim}
class Num { public int val; }

public class ReturnTest {
    public static void main(String[] args) {
        Num a = new Num(); a.val = 1;
        Num b = new Num(); b.val = 2;
        b = process(a, b);
        System.out.println(a.val + "," + b.val);
    }

    public static Num process(Num x, Num y) {
        x.val = y.val;
        y = new Num();
        y.val = 3;
        return y;
    }
}
\end{verbatim}
Choose the most correct answer. 
\begin{itemize}
\item 0) `1,2`

\item 1) `2,3`

\item 2) `2,2`

\item 3) `1,3`

\end{itemize}
\item (questionId: 103357) Which of the following lines of code, if executed independently, will result in a runtime exception? (Choose all that apply)
\begin{verbatim}
// Assume all necessary imports from java.time and java.time.temporal
\end{verbatim}
Choose all the correct answer.\begin{itemize}
\item 0) `LocalDate.of(2025, 13, 1);`

\item 1) `Duration.between(LocalDate.now(), LocalDateTime.now());`

\item 2) `Period.of(1, 1, 1).plus(Duration.ofHours(1));`

\item 3) `LocalTime.now().truncatedTo(ChronoUnit.DAYS);`

\item 4) `Period.ofMonths(12).normalized();`

\end{itemize}
\item (questionId: 100021) Consider the following directory structure and files:
\begin{verbatim}
/project
    /src
        /com
            /example
                MyClass.java
    /bin
\end{verbatim}
The file \verb|MyClass.java| contains:
\begin{verbatim}
package com.example;

public class MyClass {
    public static void main(String[] args) {
        System.out.println("Running MyClass");
    }
}
\end{verbatim}
You are currently in the \verb|/project| directory. Which sequence of commands will successfully compile and run \verb|MyClass|?
Choose the most correct answer. 
\begin{itemize}
\item 0) \begin{verbatim}
javac src/com/example/MyClass.java
java -cp src com.example.MyClass
\end{verbatim}

\item 1) \begin{verbatim}
javac src/com/example/MyClass.java
java -cp bin com.example.MyClass
\end{verbatim}

\item 2) \begin{verbatim}
javac -d bin src/com/example/MyClass.java
java -cp bin com.example.MyClass
\end{verbatim}

\item 3) \begin{verbatim}
javac -d bin src/com/example/MyClass.java
java com.example.MyClass
\end{verbatim}

\end{itemize}
\item (questionId: 100525) Which of the following code snippets will compile successfully? (Choose all that apply)
Choose all the correct answer.\begin{itemize}
\item 0) \begin{verbatim}
short s = 10;
s = s + 5;
\end{verbatim}

\item 1) \begin{verbatim}
char c = 'a';
c += 5;
\end{verbatim}

\item 2) \begin{verbatim}
final byte b1 = 10;
final byte b2 = 20;
byte b3 = b1 + b2;
\end{verbatim}

\item 3) \begin{verbatim}
float f = 1.0f;
double d = f;
\end{verbatim}

\end{itemize}
\item (questionId: 100623) Which of the following lines will compile without errors? (Choose all that apply)
Choose all the correct answer.\begin{itemize}
\item 0) \verb|Integer i = new Integer(null);|

\item 1) \verb|Double d = null; double d2 = d;|

\item 2) \verb|Byte b = 25; |

\item 3) \verb|Short s = new Short((short)10);|

\item 4) \verb|long l = new Integer(100);|

\end{itemize}
\item (questionId: 101222) Examine the following code. What is the result?
\begin{verbatim}
public enum Operation {
    PLUS {
        public double apply(double x, double y) { return x + y; }
    },
    MINUS {
        public double apply(double x, double y) { return x - y; }
    };
    public abstract double apply(double x, double y);
}

class Test {
    public static void main(String[] args) {
        System.out.println(Operation.PLUS.apply(5, 3));
    }
}
\end{verbatim}
Choose the most correct answer. 
\begin{itemize}
\item 0) `8.0`

\item 1) The code fails to compile because an enum cannot be `abstract`.

\item 2) The code fails to compile because `apply` is not defined for the `Operation` enum itself.

\item 3) The code fails to compile because an enum constant cannot provide a method implementation.

\end{itemize}
\item (questionId: 100328) Which of the following code snippets will fail to compile due to issues with comment syntax? (Choose all that apply)
Choose all the correct answer.\begin{itemize}
\item 0) \begin{verbatim}
int x = 10; //* A special comment */
\end{verbatim}

\item 1) \begin{verbatim}
String s = "This contains a comment end: */";
\end{verbatim}

\item 2) \begin{verbatim}
/* Is this /* nested comment */ valid? */
int y = 20;
\end{verbatim}

\item 3) \begin{verbatim}
// Another comment \
int z = 30;
\end{verbatim}

\end{itemize}
\item (questionId: 102622) Which of these lines causes a compilation error?\begin{verbatim}
import java.util.*;

class Mammal {}
class Primate extends Mammal {}
class Human extends Primate {}

public class Test {
    public static void main(String[] args) {
        List<? super Primate> primates = new ArrayList<Mammal>(); // Line 1
        primates.add(new Human());                                // Line 2
        primates.add(new Primate());                              // Line 3
        primates.add(new Mammal());                               // Line 4
    }
}
\end{verbatim}
Choose the most correct answer. 
\begin{itemize}
\item 0) Line 1

\item 1) Line 2

\item 2) Line 3

\item 3) Line 4

\end{itemize}
\item (questionId: 101025) What is the output of the following code?\n\begin{verbatim}
int[] a = {1, 2, 3};
int[] b = {4, 5, 6};
for (int i : a, j : b) {
    System.out.print(i + j);
}
\end{verbatim}
Choose the most correct answer. 
\begin{itemize}
\item 0) 579

\item 1) 142536

\item 2) The code does not compile.

\item 3) The code throws a runtime exception.

\end{itemize}
\item (questionId: 103558) Which of the following method declarations are valid in a concrete (non-abstract) class? (Choose all that apply)
Choose all the correct answer.\begin{itemize}
\item 0) `private final static void methodA();`

\item 1) `protected abstract void methodB();`

\item 2) `public final synchronized void methodC(String... s) {}`

\item 3) `void methodD(final int... x) {}`

\item 4) `static { System.out.println("I am not a method."); }`

\end{itemize}
\item (questionId: 101122) What is the result of attempting to compile this code?\n\begin{verbatim}
public class InvalidContinue {
    public static void main(String[] args) {
        myLabel: {
            if (true) {
                continue myLabel; 
            }
        }
    }
}
\end{verbatim}
Choose the most correct answer. 
\begin{itemize}
\item 0) It compiles successfully.

\item 1) It fails to compile because `myLabel` is not on a loop.

\item 2) It fails to compile because a label cannot be on a simple block.

\item 3) It fails to compile because of an unreachable statement.

\end{itemize}
\item (questionId: 100625) What is the output of the following code?\n\begin{verbatim}
import java.util.ArrayList;
import java.util.List;

public class Test {
    public static void main(String[] args) {
        List<Integer> list = new ArrayList<>();
        list.add(1);
        list.add(2);
        list.add(3);
        list.remove(new Integer(2));
        System.out.println(list);
    }
}
\end{verbatim}
Choose the most correct answer. 
\begin{itemize}
\item 0) \verb|[1, 2]|

\item 1) \verb|[1, 3]|

\item 2) \verb|[2, 3]|

\item 3) An \verb|IndexOutOfBoundsException| occurs.

\end{itemize}
\item (questionId: 101725) What is the output of this code? This tests understanding of `this` within inner classes.\n\begin{verbatim}
public class Outer {
    String name = "Outer";

    class Inner {
        String name = "Inner";
        void printNames() {
            System.out.println(name);
            System.out.println(this.name);
            System.out.println(Outer.this.name);
        }
    }

    public static void main(String[] args) {
        new Outer().new Inner().printNames();
    }
}
\end{verbatim}
Choose the most correct answer. 
\begin{itemize}
\item 0) Inner\nInner\nOuter

\item 1) Outer\nOuter\nOuter

\item 2) Inner\nOuter\nOuter

\item 3) The code fails to compile.

\end{itemize}
\item (questionId: 100223) The classpath is set to `-cp dirA:dirB`. `dirA` contains `com/test/Tool.class` version 1. `dirB` contains `com/test/Tool.class` version 2. A program uses `com.test.Tool`. Which version of the class will be loaded by the JVM?
Choose the most correct answer. 
\begin{itemize}
\item 0) Version 1 from `dirA`.

\item 1) Version 2 from `dirB`.

\item 2) A compilation error will occur.

\item 3) A runtime error will occur due to the conflict.

\end{itemize}
\item (questionId: 101428) Which of these method calls can throw a `StringIndexOutOfBoundsException`? (Choose all that apply)
\begin{verbatim}
StringBuilder sb = new StringBuilder("abc");
\end{verbatim}
Choose all the correct answer.\begin{itemize}
\item 0) `sb.delete(1, 4);`

\item 1) `sb.insert(4, "d");`

\item 2) `sb.replace(0, 5, "x");`

\item 3) `sb.setCharAt(3, 'd');`

\end{itemize}
\item (questionId: 102824) What is the result of attempting to compile this code?
\begin{verbatim}
public class Test {
    public static void main(String[] args) {
        throw new String("This is an error");
    }
}
\end{verbatim}
Choose the most correct answer. 
\begin{itemize}
\item 0) It compiles, but throws a `ClassCastException` at runtime.

\item 1) It compiles, but throws a `RuntimeException` at runtime.

\item 2) It compiles, but throws an `Error` at runtime.

\item 3) It does not compile.

\end{itemize}

\end{enumerate}

\end{document}